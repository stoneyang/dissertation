%# -*- coding:utf-8 -*- 
%-------------------------------------------------------------------------%
%   Latex 博士论文模板使用说明.
%   建议使用miktex2.1最大安装编译此模板
%   本模板使用ctex.org的cjk版本 和aloft 修改后的gb_aloft.cap,
%   如果安装的是miktex包自带的cjk 请用srcbook 替换 book
%=========================================================================%
%%%%%%%%%%%%%%%%%%%%%%%%%%%%%%%%%%%%%%%%%%%%%%%%%%%%%%%%%%%%%%%
%文件头部定义
%%%%%%%%%%%%%%%%%%%%%%%%%%%%%
\documentclass [12pt,a4paper,openany,twoside] {article}
%\documentclass [12pt,a4paper,openany,twoside,draft] {book}
%============================ 引用的宏包 ==================================%
\usepackage{CJKutf8}
\usepackage{url}
\usepackage{ccmap}
\begin{document}
\begin{CJK*}{UTF8}{song}
\title{中科院自动化所博士/硕士论文\LaTeX 模板}
\author{张兆翔\quad \quad 楼建光\\模式识别国家重点实验室}
\date{2007年4月}
\maketitle
\section{关于模板}
该模板原由楼建光撰写,张兆翔于2007年4月根据最新要求进行了修改。
对于本模板中存在的不尽善尽美之处,欢迎大家使用过程中提出来,并联系我。\\
电子邮件:\tt{zxzhang@nlpr.ia.ac.cn}
\section{前言}
毕业论文撰写是每一位毕业在即的博士生和硕士生的一项大事,大家常常为论文的格式和规范伤透脑筋,
研究生部也对大家递交的不同格式的论文感到很难管理。鉴于此,我们设计了采用\LaTeX
博士或硕士论文模板。为什么选择\LaTeX ?
\begin{itemize}
    \item 提供专业级的排版设计,如同印刷好的一样;
    \item 方便强劲的数学公式排版;
    \item 方便的生成目录、引用、参考文献、索引等复杂结构;
    \item 使得文章具有良好的结构;
    \item 让作者放精力到写作上,而不是排版上;
    \item 完全免费的\LaTeX{} 软件,适应多种操作系统;
    \item \LaTeX{} 还是IEEE Press等著名学术机构的指定论文排版工具。
\end{itemize}
与Word比较:
\begin{itemize}
    \item 采用Word制作论文时,当你修改文字内容后,作者得自己重新调整图、表位置。\LaTeX
    中,作者不用关心图、表放置的位置,论文的格式由模板决定,作者只需要关心文章的内容和结构。
    \item 在Word中,往往由于论文中使用的图、表、公式都是以COM对象存放的,系统效率很低,经常
    在操作图、表比较多的论文时,导致系统死机或运行非常慢。如果把论文分成几个doc文档,则又无法
    自动生成目录等结构。在\LaTeX 中,系统效率与文章的长短以及图、表数量无关,作者可以按照文章结构
    随意分割文件。
\end{itemize}

当然,\LaTeX{} 也有其缺点,它不是所见即所得的。
\section{论文的结构}
论文的结构分为:
\begin{itemize}
    \item[一、] 封面部分(cover)---中文封面、英文封面、独创性声明
    \item[二、] 前言部分(Preface)---中文摘要(c\_abstract)、英文摘要(e\_abstract)
    \item[三、] 目录(contents)---自动生成的目录
    \item[四、] 正文部分(body)---第一章(chap01)、第二章(chap02)等等
    \item[五、] 参考文献
    \item[六、] 攻读博士学位期间的研究成果
    \item[七、] 致谢
\end{itemize}
作者只需要将相应的内容输入到相应文件中,不需要再调整格式。

\clearpage
\end{CJK*}
\end{document}
