%# -*- coding:utf-8 -*-
%=========================================================================%
%   LaTeX File for phd thesis of Institute of Automation, CAS
%-------------------------------------------------------------------------%
%   Revised by J. G. Lou (jglou@nlpr.ia.ac.cn)
%   中文摘要页
%-------------------------------------------------------------------------%
\chapter*{摘\;\;\;要 \markboth{摘\;\;要}{摘\;\;要}}

\vspace{0.5cm} \wuhao
%以下为摘要的正文

在本文中,我们研究了经皮冠状动脉介入术仿真训练系统的关键技术。研究工作的重点集
中在与经皮冠状动脉介入术这一被大量采用的治疗冠心病的治疗技术过程相关的人体解剖
环境模型的生成方面,尤其关注基于CTA体数据的血管系统模型的建立和心脏模型的建立。
在传统的手术操作和学习过程、尤其是学习过程中,仍存在着一些不足,例如:见习机会
的稀缺,活体动物的高成本,以及培训考核中评估意见的主观性等。仿真训练系统为解决
这些问题提供了可能性。本文的贡献在于对这种仿真训练系统的关键技术的研究。

本文的关注点在于与经皮冠状动脉介入术相关的血管系统等人体解剖结构的几何建模和物
理建模。在这一能够表达出恰当物理特性的解剖环境模型中,虚拟手术工具可以在其中进
行类似被仿真手术中的各种基本操作,这样,就能够提高训练的真实性,进而为受训人员
的技能评估提供了客观依据,最终保证了训练的质量水平。

基于区域的图像分割方法和基于水平集的图像分割方法被用于CTA体数据中相关人体解剖结
构的分割提取。面绘制的渲染方法被用于被提取体素的可视化。然后,我们对所得几何模
型的数据进行优化,改善解剖结构的可视化效果。接着,我们对不同的解剖结构的物理特
性分别进行分析和建模,实现一系列仿真手术操作的视觉特效,增强了相应器官和主要手
术操作的真实性。最后,我们在这些工作成果的基础上,进行手术过程的状态建模,为整
个操作过程的记录和评估提供基础支持。在这些成果的基础上,我们设计并开发了经皮冠
状动脉介入术仿真训练系统样机(VitroCathIA)来验证我们的工作。

经皮冠状动脉介入术仿真训练是虚拟现实技术在医学教学领域、尤其是介入心脏病学科教
育的应用。本文针对用经皮介入冠脉导管术仿真系统的关键技术进行了深入的研究,涉及
系统工程、材料工程、机器人工程、图像处理、计算机图形学、计算机科学、生物医学工
程、医学等学科。本文将从以下几个方面论述作者的主要学术工作和知识贡献:
\begin{enumerate}
    \item 提出了一种基于XXX的血管树提取方法。该方法以CTA为处理对象,经过XXX,获
    得了满足本文工作所需的表示血管区域的影像信息。
    \item 提出了一种基于XXX的血管树重建方法。该方法以从CTA中所得到的影像信息为处
    理对象,经过XXX,获得了血管树的可视化模型。
    \item 提出了一种基于XXX的心脏提取方法。该方法以CTA为处理对象,经过XXX,得到
    了满足本文工作要求的表示心脏区域的影像信息。
    \item 提出了一种基于XXX的心脏重建方法。该方法以从CTA中所得到的影像信息为处理
    对象,经过XXX,获取了心脏的可视化模型。
    \item 研究并实现了人体解剖环境的可视化模型的重建。采用XXX,对XXX进行XXX,得
    到了为手术仿真训练系统的实现提供了基础支持。
    \item 深入研究了血管和心脏的生理和物理特性,初步实现了可视化模型的物理特性,
    为虚拟手术工具与解剖环境的交互提供了进一步保障。
    \item 研究了经皮冠状动脉介入术的完整流程。初步实现了手术仿真的状态机模型,为
    整个手术仿真系统提供了关键设施。
    \item 设计并开发了经皮冠状动脉介入术仿真系统的软件系统。
\end{enumerate}
总的说来,本文在针对经皮冠状动脉介入术仿真系统的关键技术方面作了有益的探索。
\\
\\
\\
\noindent \textbf{关键词:} 虚拟现实,手术仿真训练,医学可视化,医学影像处理,
碰撞检测,形变模型,组织建模,经皮冠状动脉介入术
