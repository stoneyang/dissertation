%# -*- coding:utf-8 -*- 
%=========================================================================%
%   LaTeX File for phd thesis of Institute of Automation, CAS
%-------------------------------------------------------------------------%
%   Revised by J. G. Lou (jglou@nlpr.ia.ac.cn)
%   中文摘要页
%-------------------------------------------------------------------------%
\chapter*{摘要 \markboth{摘要}{摘要}}

\vspace{0.5cm} \wuhao
%以下为摘要的正文

动态图像序列理解就是要赋予计算机类似于人一样的观察和理解
动态场景的视觉能力,通过对图像数据的分析与处理来获取对
动态景物中运动物体行为及其相互关系的高层次语义上的解释,
实现从图像空间的数值描述到概念空间的语义描述的转换。传统的
计算机视觉研究工作主要集中在恢复场景的三维几何结构,计算摄像机
的运动和象素的运动信息等,很少涉及到图像的高层语义信息。近年来,在
高层语义与行为理解方面的工作受到了越来越多的关注。但是,在这个领域
中还存在很多理论与技术问题需要解决。

本文针对路面交通场景动态图像序列的语义理解进行了深入的研究,涉及到了许多
动态图像语义理解的基本问题,包括摄像机标定、运动检测和分割、目标定位、时空推理、
场景恢复与表示、行为分析和建模、语义理解等等。在本文中,主要的工作和贡献有:
\begin{enumerate}
    \item 提出了一种简单方便的适用于路面交通场合的摄像机参数求取方法;
    改进了运动检测的算法模块,使得其对光照有更好的鲁棒性;还提出了一种车
    辆定位的算法。
    \item 提出了利用一种改进的Kalman滤波器的车辆视觉跟踪算法,包括对
    车辆的运动建立动态模型,和与其他常用跟踪滤波器的比较。实验证明,由于
    引入了合理的正交性约束,使得这种改进的Kalman滤波器能够在复杂运动下具有更
    好的跟踪能力。
    \item 发展了一种高层语义理解的框架,引入了概念空间来实现从定量的几何
    描述到语义概念之间的映射,从而在传统计算机视觉和高层语义理解与推理之间架起一座桥梁。

    \item 深入研究人类运动概念的抽象层次,非常合理地区分了普遍度(Generality)和复杂
    度(Complexity)两种人类语义概念中的不同角度的抽象层次,并在不同的处
    理过程中对其建立模型。
    \item 提出了采用时间间隔模型来对行为建模和识别的方法。这种方法可
    以非常方便地为多个运动目标之间的交互行为建模,并利用建立的模型进行行为识别。
    最后还利用简单的语法规则,自动产生对语义概念的自然语言描述。
    \item 研究并实现了一个基于三维线框模型的路面交通监控与行为理解系统平台,
    平台工作在利用普通PC机上,已经能够在PIV 1.7G的PC机256M内存配置下近乎实
    时(十几帧每秒)地跟踪单辆车,初步实现对特定场景下的物体行为作出简单的
    自然语言描述。
\end{enumerate}
总的说来,本文在针对一类特定场景(路面交通场景)的动态图像序列理解作了有益的探索。

\noindent \textbf{关键词:} 动态图像序列理解,视觉跟踪,行为分析与识别,自然语言描述
