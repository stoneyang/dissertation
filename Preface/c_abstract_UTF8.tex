%# -*- coding:utf-8 -*- 
%=========================================================================%
%   LaTeX File for phd thesis of Institute of Automation, CAS
%-------------------------------------------------------------------------%
%   Revised by J. G. Lou (jglou@nlpr.ia.ac.cn)
%   中文摘要页
%-------------------------------------------------------------------------%
\chapter*{摘要 \markboth{摘要}{摘要}}

\vspace{0.5cm} \wuhao
%以下为摘要的正文

受人类现代生活压力大、节奏快等因素的影响,心血管疾病的发病率和致死率逐年攀升。
心血管疾病已经对人类健康造成了严重的危害。随着现代医学的发展,介入式心脏导管术
成为治疗这类疾病的有效手段。与传统手术相比,介入式心脏导管术具有创口小、痛苦少
、住院时间短、恢复快等优势。

然而,相较于传统手术,介入式导管术仍具有一定的局限性,一方面是手术的难度。在整
个介入式手术过程中,医生无法直接通过肉眼观察手术器械的精确位置、而只能通过X光
成像来判断器械的当前位置并估计下一步的运动方向,这给医生的手眼协调造成了障碍;
手术工具只能在血管管腔内进行旋转和进退运动;医生从手术工具获得的触感有限,不利
于判断手术工具是否与血管壁发生接触、甚至碰撞。

另一方面是培训的问题。为了得到充分的训练,见习人员的培训时间较长,能否在手术室
中观察真实手术完全取决于有没有相关病症的病人,使这样的机会变得非常珍贵,而目前
不断上涨的卫生医疗支出又加剧了这个问题,使得临床学习时间急剧减少;另外,对医生
手术技术的评估缺乏客观的、标准的方法和流程,整个评估由经验丰富的高级医务人员进
行观察和判断被评估者的水平,这存在着主观性,不利于训练质量水平的控制。

关于介入手术的训练方法,医学界尚未对何种训练方式更为高效取得一致看法。为了克服
这些问题,欧美的许多在相关医学技术领域处于领先地位的研究机构先后建立了专门的训
练中心,用于培训相关人员,其训练模型包括静态模型和动态模型。这些模型提供了真实
的手术工具、显示器、以及塑料制成的人体组织模型。然而,这些模型仍旧缺乏足够的真
实性,而且,我们还是无法客观评估在这些模型上进行训练的见习人员的技术水平。

目前,最为有效的训练方式当属活体动物训练。在活体动物上进行介入式手术训练,受训
人员可以获得尽可能丰富的解剖结构,以及由每个操作所产生的相应生理反应。虽然这些
都与人体不尽相同,这种训练方法毕竟为受训人员提供了真实性。然而,在活体动物上进
行医学和生物实验成本高昂而且存在争议。

基于计算机虚拟现实技术的手术仿真训练技术应运而生,它可以为受训人员提供与手术相
关的触觉、视觉等感官信息,让受训人员可以像在进行真正手术一样完成训练,而无需借
助过多的真实物料、无需依赖所在单位的时间安排。训练的整个过程可由计算机进行记录
并评估,为实现训练评估的客观性和标准化提供了技术支持。

本文针对用经皮介入冠脉导管术仿真系统的关键技术进行了深入的研究,涉及系统工程、
材料工程、机器人工程、计算机科学、生物医学工程、医学等学科。本文将从以下几个方
面论述本人的主要学术工作和知识贡献:
\begin{enumerate}
    \item 提出了一种基于XXX的血管树提取方法。该方法以CTA为处理对象,经过XXX,获
    得了满足本文工作所需的表示血管区域的影像信息。
    \item 提出了一种基于XXX的血管树重建方法。该方法以从CTA中所得到的影像信息为处
    理对象,经过XXX,获得了血管树的可视化模型。
    \item 提出了一种基于XXX的心脏提取方法。该方法以CTA为处理对象,经过XXX,得到
    了满足本文工作要求的表示心脏区域的影像信息。
    \item 提出了一种基于XXX的心脏重建方法。该方法以从CTA中所得到的影像信息为处理
    对象,经过XXX,获取了心脏的可视化模型。
    \item 研究并实现了人体解剖环境的可视化模型的重建。采用XXX,对XXX进行XXX,得
    到了为手术仿真训练系统的实现提供了基础支持。
    \item 深入研究了血管和心脏的生理和物理特性,初步实现了可视化模型的物理特性,
    为虚拟手术工具与解剖环境的交互提供了进一步保障。
    \item 研究了经皮介入腔内导管术的完整流程。初步实现了手术仿真的状态机模型,为
    整个手术仿真系统提供了关键设施。
    \item 设计并开发了经皮介入冠脉导管术仿真系统的软件系统。
\end{enumerate}
总的说来,本文在针对经皮介入冠脉导管术仿真系统的关键技术作了有益的探索。
\\
\\
\\
\noindent \textbf{关键词:} 虚拟现实,手术仿真训练,医学可视化,医学影像处理,
碰撞检测,形变模型,组织建模
