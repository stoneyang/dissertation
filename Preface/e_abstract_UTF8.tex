%# -*- coding:utf-8 -*- 
%=========================================================================%
%   LaTeX File for phd thesis of Institute of Automation, CAS
%-------------------------------------------------------------------------%
%   Revised by J. G. Lou (jglou@nlpr.ia.ac.cn)
%-------------------------------------------------------------------------%
%%%%%%%%%%%%%%%%%%%%%%%%%%%%%%%%%%%%%%%%%%%%%%%%%%%%%%%%%%%%%%%%%%%%%%%%%

\chapter*{ \markboth{英文摘要}{英文摘要}}

\vspace*{-1.5cm}

\begin{center} \xiaoerhao \textsf{\textbf{Research on Key Technologies of Percutaneous Coronary Intervention Simulation}}\\
\xiaosihao Author: Fan Yang \\ Directed by Zengguang Hou\\
\xiaoerhao Abstract\\
\end{center}

\fontsize{12pt}{12pt}\selectfont \vspace{0.5cm}

Semantic interpretation of dynamic image sequences tries to make it possible that
the computer can automatically interpret the motions and behaviors of the tracked
objects by the analysis of the image sequences captured by cameras from wide-area,
real-world scenes in natural conditions. Traditional computer vision research mainly
focused on recovering the geometry of the scene (structure from motion), the camera
motion (ego-motion), and the motion of the pixels themselves (such as optical flow).
In recent years, semantic interpretation of image and video has become an active
topic in computer vision.

In this thesis, we study the semantic interpretation of image sequences from traffic
scenes which involves a lot of basic problems in computer vision e.g. motion
detection, target localization, spatial-temporal reasoning, behavior analysis and
semantic interpretation, etc. The main contributions of this thesis include
following issues:
\begin{enumerate}
    \item We propose a simple but convenient method for camera calibration in
    traffic scenes, an improved motion detection with less sensitivity to lighting,
    an efficient and robust vehicle localization algorithm.
    \item We propose a modified extended Kalman filter incorporated with a precise
    kinematics model for visual vehicle tracking. By being combined with an additional orthogonality
    condition, the filter has less sensitivity to the time varying model of system. Experiments
    show that the filter has a good performance when the tracked car is in a complex motion.
    \item In this thesis,a framework for semantic interpretation of vehicles and pedestrians'
    actions is proposed for practical applications in visual traffic surveillance.  We introduce
    a conceptual space to bridge the gap between low-level processing which is quantitative
    and high-level processing where information is handled by qualitative means.
    \item From human's \`mental experiences, there are two aspects of abstraction: ``generality'' and
    ``complexity". We deal with them in two deferent computational stages named ``conceptual process'' and ``symbolic process'' to
    simplify the modelling and inference for a computational aim.
    \item We propose a new interval-based model of action and a temporal analyzer to model and
    recognize the targets' behaviors in traffic scenes. A single object's behaviors and its
    interactions with other objects can be handled in the same framework. Finally, some of the recognized
    actions can be selected and translated into natural language descriptions by some simple grammar
    rules.
    \item We develop a demo platform for further research which can work at
    speed of 17 frames per second on a computer with PIV 1.7G CPU and Windows
    operating system. Now, the system can give some simple semantic interpretations
    of vehicle's behaviors.
\end{enumerate}

In a word, in this thesis, we have made a lot of fruitful attempts and significant
progresses on our surveillance system.

\noindent \textbf{Key Words:} semantic interpretation of dynamic image sequences,
visual tracking, behavior analysis and recognition, natural language description
