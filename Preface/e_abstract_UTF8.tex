%# -*- coding:utf-8 -*- 
%=========================================================================%
%   LaTeX File for phd thesis of Institute of Automation, CAS
%-------------------------------------------------------------------------%
%   Revised by J. G. Lou (jglou@nlpr.ia.ac.cn)
%-------------------------------------------------------------------------%
%%%%%%%%%%%%%%%%%%%%%%%%%%%%%%%%%%%%%%%%%%%%%%%%%%%%%%%%%%%%%%%%%%%%%%%%%

\chapter*{ \markboth{英文摘要}{英文摘要}}

\vspace*{-1.5cm}

\begin{center} \xiaoerhao \textsf{\textbf{Research on Key Technologies for Percutaneous Coronary Intervention Simulation}}\\
\xiaosihao Author: Fan Yang \\ Directed by Zeng-guang Hou\\
\xiaoerhao Abstract\\
\end{center}

\fontsize{12pt}{12pt}\selectfont \vspace{0.5cm}

Due to the high pressure and fast pace of modern life, the rate of incidence 
and mortality of cardiovascular diseases are rising in the past half a century. 
Cardiovascular diseases had been the evil of human health and well-being. 
Thanks to the development of modern medicine during the very period, 
interventional procedures emerged and had become the effective therapies for 
the diseases. Compared to the traditional procedures, the noval interventional 
ones have the advantage of smaller incisions, slighter pains, shorter period 
in hospital, and faster recovery. 

However, in respect to the tradition techniques, the interventional procedures 
have their limitations. First, the techniques are hard to learn. During the 
course of the procedure, due to the tool is delivered into the cavity through 
a small incision, one cannot directly observe the actual location of the tool 
by his or her naked eyes, thus making the hand-eye coordination difficult. The 
only way to achieve this is to employ the fluroscopy, which in turn introduces 
the risk of radiation, for both the clinician and the patient. Due to the small 
size of the incision, the clinician can only manipulate the tool in its axial 
direction, only translational or rotational motion can be achieved. The haptic 
feelings recevied from the tool too weak to judge whether the tool had contact 
on the vessel wall.

On the other hand, there are issues related to the training itself. The residency 
requires several years to acquire the skills required to perform the procedures. 
Ideally, in the residency course, the students observe and participate enough 
operations and then they become the competent clinicians. But the high cost of 
health care spoil this ideal settings by reducing the training period in operation 
room. In addition, we lack the objective and standardized methods in the 
evaluation and validation, the whole process is subjectively judged by the senior 
clinicians. It is not good for the control of quality of the training.

There is no consensus on the most effective way of training in the medicine 
academia. To undertake these problems, lots of leading medical institutions in 
America and Europe had established their dedicated surgical training centers 
for the training of the complex surgical techniques. The training models can be 
catagorized into inanimate models and animate models. Unfortunately, the 
performance of these models are poor for the lack of reality. Moreover, it is 
still hard to exclude the subjectvity during the evaluation.

The virtual-reality based surgical simulation methods emerged as a revolutionary 
alternative for the traditional trainining. The simulators let the trainee see, 
touch, and feel the virtual scene generated by the computer, and perform their 
tasks with the haptic interface. What's more important, the simulators do have 
provide the support for the standardization of the evaluation of the training.

In this thesis, we study the key technologies for the surgical simulators for the 
percutaneous coronary intervention procedures, which involves many disciplines, 
such as, system engineering, material engineering, robotic engineering, computer 
science, biomedical engineering, and medicine, etc. The main contributions of 
this thesis include following issues:
\begin{enumerate}
    \item We propose a simple but convenient method for camera calibration in
    traffic scenes, an improved motion detection with less sensitivity to lighting,
    an efficient and robust vehicle localization algorithm.
    \item We propose a modified extended Kalman filter incorporated with a precise
    kinematics model for visual vehicle tracking. By being combined with an additional orthogonality
    condition, the filter has less sensitivity to the time varying model of system. Experiments
    show that the filter has a good performance when the tracked car is in a complex motion.
    \item In this thesis,a framework for semantic interpretation of vehicles and pedestrians'
    actions is proposed for practical applications in visual traffic surveillance.  We introduce
    a conceptual space to bridge the gap between low-level processing which is quantitative
    and high-level processing where information is handled by qualitative means.
    \item From human's \`mental experiences, there are two aspects of abstraction: ``generality'' and
    ``complexity". We deal with them in two deferent computational stages named ``conceptual process'' and ``symbolic process'' to
    simplify the modelling and inference for a computational aim.
    \item We propose a new interval-based model of action and a temporal analyzer to model and
    recognize the targets' behaviors in traffic scenes. A single object's behaviors and its
    interactions with other objects can be handled in the same framework. Finally, some of the recognized
    actions can be selected and translated into natural language descriptions by some simple grammar
    rules.
    \item We develop a demo platform for further research which can work at
    speed of 17 frames per second on a computer with PIV 1.7G CPU and Windows
    operating system. Now, the system can give some simple semantic interpretations
    of vehicle's behaviors.
\end{enumerate}

In a word, in this thesis, we have made a lot of fruitful attempts and significant
progresses on our simulation system.
\\
\\
\\
\noindent \textbf{Key Words:} virtual reality, surgical simulation, medical 
visualization, medical imaging, collision detection, deformable models, tissue modeling
