%# -*- coding:utf-8 -*-
%=========================================================================%
%   LaTeX File for phd thesis of Institute of Automation, CAS
%-------------------------------------------------------------------------%
%   Revised by J. G. Lou (jglou@nlpr.ia.ac.cn)
%-------------------------------------------------------------------------%
%%%%%%%%%%%%%%%%%%%%%%%%%%%%%%%%%%%%%%%%%%%%%%%%%%%%%%%%%%%%%%%%%%%%%%%%%

\chapter*{致\ 谢\markboth{致\ 谢}{致\ 谢}}
\addcontentsline{toc}{chapter}{致谢}



首先,我要向我的导师侯增广研究员表示最衷心的感谢和敬意。感谢老师给我加入机器人
研究组工作的机会。感谢老师在我学习期间给予我的指导和帮助。老师渊博的学识,对学
术发展的敏锐眼光,对科研事业的执着追求和对工作的热忱,以及对我们研究生的关心和
爱护,都深深地鼓舞和影响了我。所有这些,都令学生受益终生!

感谢复杂系统管理与控制国家重点实验室机器人研究组的谭民研究员等各位老师为我们营
造的这样一个良好的学习环境。感谢同组已毕业的同学吉程博士,与他的讨论激发了我的
创意和兴趣;感谢同组同届的同学米韶华和陈翼雄,共同攻读博士学位的“苦难岁月”让
我们成为了战友;感谢同组的程龙博士等研究人员和同学对我的帮助和支持。在机器人研
究组学习和工作的时光,将是我个人学习生涯中最难以忘怀的。

感谢上海市胸科医院的方唯一教授和曲新凯博士,感谢他们在百忙中抽出宝贵的时间为我
们耐心地讲解和演示经皮介入冠状动脉导管术,热心回复我们在工作中遇到的各种医学方
面的问题,并为我们提供了宝贵的医学资料。

感谢开源软件社区的每一位黑客,感谢他们在智慧上的无私分享和热情帮助,这使我从中
学到了许多技巧和技术以及这些背后的可贵精神。

感谢中国科学院自动化研究所研究生部的各位工作人员为我在研究所学习和工作期间提供
的咨询和帮助。

最深的感谢送给我的父母,感谢他们对我孜孜不倦的教诲和无限的宽容与鼓励。感谢我的
妹妹,家庭的开心果,感谢与她共同成长成人的岁月,感谢她带给我们的每一次开怀大笑。
最后,我要感谢我的妻子,感谢她给我无尽的爱和支持。
\\
\\
\\
\\
\hspace*{10.8cm}2013年10月
