%# -*- coding:utf-8 -*- 
%=========================================================================%
%   LaTeX File for phd thesis of Institute of Automation, CAS
%-------------------------------------------------------------------------%
%   Revised by J. G. Lou (jglou@nlpr.ia.ac.cn)
%-------------------------------------------------------------------------%
%%%%%%%%%%%%%%%%%%%%%%%%%%%%%%%%%%%%%%%%%%%%%%%%%%%%%%%%%%%%%%%%%%%%%%%%%

\chapter*{致\ 谢\markboth{致\ 谢}{致\ 谢}}
\addcontentsline{toc}{chapter}{致谢}



至此学位论文完成之际,请允许我感谢我的导师侯增广研究员。老师渊博的学识,对学术
发展的敏锐眼光,对科研事业的执着追求和对工作的热忱,以及对我们青年学生的关心和
爱护,都深深地鼓舞和影响了我。所有这些,都令学生受益终生!

感谢复杂系统管理与控制国家重点实验室机器人研究组的各位老师为我们营造的这样一个
良好的学习环境。感谢同组已毕业的同事吉程博士和徐飞硕士,与他们的讨论激发了我的
创意和兴趣;感谢同组同届的同事米韶华和陈翼雄,与他们共同度过这攻读博士学位的
“苦难岁月”让我们成了拥有某种类似战友的感情;感谢同组的其他同事,感谢他们的帮
助和支持。

感谢上海市胸科医院的方唯一教授和曲新凯博士,感谢他们为我们;

感谢开源软件社区的每一位黑客,

感谢中国科学院自动化研究所研究生部的各位工作人员在为我提供的咨询和帮助。

最深的感谢送给我的父母,感谢他们孜孜不倦的教诲,如果没有那时无限的宽容与鼓励,
恐怕也就没有我后来的任何一次进步。感谢我的妹妹,家庭的开心果,感谢与她共同长大
成人的日子,感谢她带给我们的每一次开怀大笑。最后,我要感谢我的女友、我的妻子,
感谢她给我无尽的爱和支持,为了完成这项工作,我们错过了许多享受二人时光的周末。





导师渊博的学识,对学术研究的敏锐眼光,以及对事业的执著追
求和对工作的热情,都深深地鼓舞和影响了我,令我受益终生!

感谢模式识别国家重点实验室的各位老师和同学,是他们共同创造了一个良好的学习氛围,
能和这些志同道合的朋友们愉快共事,对我而言是一段非常
宝贵而且难忘的人生经历,更何况得到了他们大量热情无私的帮助。

我要感谢我的父母,他们为了让我完成学业,拖着病老的身体终日劳作在农田上。最后,我还要感
谢我的女朋友,她一直在背后支持着我,鼓励我前进。他们的爱给了我克服困难的勇气和积极进取的
热情。

\hspace*{10.8cm}2003年4月
