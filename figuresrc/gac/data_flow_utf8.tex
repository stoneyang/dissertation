%# -*- coding:utf-8 -*-
\begin{tikzpicture}[scale=.37]

\draw [black,thick,rounded corners] (-3,0) rectangle (3,2);            % binary threshold
\draw [black,thick,rounded corners] (-3,3) rectangle (3,5);  % GAC

\draw [black,thick,rounded corners] (-8,7) rectangle (-2,9); % fast marching
\draw [black,thick,rounded corners] (-8,13) rectangle (-2,15); % thresholding
\draw [black,thin,dashed] (-8.5,6.5) rectangle (-1.5,15.5);
\node [above right] at (-8.8,15.5) {\scriptsize \fs \bf 初始水平集};

\draw [black,thick,rounded corners] (2,7) rectangle (8,9);   % sigmoid
\draw [black,thick,rounded corners] (2,10) rectangle (8,12); % gradient
\draw [black,thick,rounded corners] (2,13) rectangle (8,15); % curvature anisotropic diffusion
\draw [black,thin,dashed] (1.5,6.5) rectangle (8.5,15.5);
\node [above right] at (5.3,15.5) {\scriptsize \fs \bf 特征图像};

\draw [black,thick,rounded corners] (-3,17) rectangle (3,19); % raw input

\node [above right] at (-2.4,0.3) {\scriptsize \fs \bf 二值阈值滤波};
\node [above right] at (-2.4,3.3) {\scriptsize \fs \bf 测地活动轮廓};

\node [above right] at (-6.75,7.3) {\scriptsize \fs \bf 快速行进};
\node [above right] at (-6.75,13.3) {\scriptsize \fs \bf 阈值滤波};

\node [above right] at (2.0,7.3) {\scriptsize \fs \bf 亮度的非线性映射};
\node [above right] at (2.7,10.3) {\scriptsize \fs \bf 梯度幅值计算};
\node [above right] at (2.0,13.3) {\scriptsize \fs \bf 曲率各向异性扩散};

\node [above right] at (-2.05,17.3) {\scriptsize \fs \bf 原始体数据};

\draw [<-,thick] (0,2) -- (0,3);

\draw [<-,thick] (0,5) -- (0,6);
\draw [thick] (-5,6) -- (5,6);
\draw [thick] (-5,6) -- (-5,7);
\draw [thick] (5,6) -- (5,7);

\draw [<-,thick] (-5,9) -- (-5,13);
\draw [<-,thick] (5,9) -- (5,10);

\draw [<-,thick] (5,9) -- (5,10);
\draw [<-,thick] (5,12) -- (5,13);

\draw [<-,thick] (-5,15) -- (-5,16);
\draw [<-,thick] (5,15) -- (5,16);
\draw [thick] (-5,16) -- (5,16);
\draw [thick] (0,16) -- (0,17);

\end{tikzpicture} 