%# -*- coding:utf-8 -*-

In fighting the coronary heart diseases, percutaneous coronary intervention is proved to be a powerful and reliable clinical procedure in the modern catheterization labs all over the world. %
Due to its minimally invasive characteristics, the procedure must be performed in the image-guided way, which makes this important skill very difficult to learn.
To make the learning more accessible, a computer-aided surgical simulator is planned to be implemented in our lab.
%The simulation system will be built upon two integral components: one is the virtual anatomic environment, and the other is the virtual surgical tools.
In implementing the virtual anatomic environment, we aim to provide the trainee an intuitive visual effect so that the models of the organs bear resemblance to their counterpart of the human's. %
Besides the blood vessels per se, the surrounding organs seen in the real surgery also need to be visualized during the simulation.
The heart is undoubtedly the most critical one among them.
The segmentation of the heart is a challenging task because of the noisy and indistinct boundaries of the heart in the images due to the natural heart beating during the image acquisition. %
In this paper, an approach based on the active contours method is developed to fulfill this job.
%The raw data is firstly smoothed and is then thresholded into a well-shaped binary images.
%Next, the preprocessed data is distributed to compute the distance map and the edge potential map, respectively.
%After that, the active contours are evolved with the destination of the boundaries of the heart.
%Finally, the resultant information is organized and rendered into a surface model.
The experimental results demonstrate the effectiveness of our approach.