%# -*- coding:utf-8 -*-
\chapter{软件系统的整合}
\label{chap9}

自从Marr在其1982年的著名论著\cite{Marr:1982}中提出计算机视觉的理
论框架以来,世界上掀起了研究计算机视觉的高潮,研究者们针对视觉的有关
问题进行了大量的工作,并取得了很大进展。但是,传统的计算机视觉研究
常常认为视觉感知过程的主要任务是物体的三维重建,例如Marr提出的一般
化圆柱模型(generalized
cylinders)\cite{Marr:1982}。从而,二十世纪八九十年代大量关于图像
序列的研究工作主要集中在恢复场景的三维几何结构(如structure from
motion),计算摄像机的运动(ego-motion)和象素的运动信息(例如 光流计算optical
flow)等,很少涉及到图像的高层语义信息。近几年来,对图像序列中的运动
理解越来越受到国际学术界的重视,甚至有人预测动态图像
序列分析和理解将成为21世纪计算机视觉的研究重点。

本文针对路面交通场景动态图像序列的语义理解进行了深入研究,涉及到许多
动态图像语义理解的基本问题,包括摄像机标定、运动检测和分割、目标定位和识别、时空推理、
场景恢复与表示、行为分析和建模、语义理解等等。
