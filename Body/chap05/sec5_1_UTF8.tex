%# -*- coding:utf-8 -*-
\section{Related Work}
\label{sec5_1}

Many researches have been done since the earliest work on extracting the centerline was proposed by Blum \cite{Blum1967}.
According to reference \cite{Ogniewicz1995}, most of these methods fall into four categories: (1) topological thinning; (2) distance transformation; (3) ``prairie fire" approach; (4) Voronoi Diagram methods. %

Topological thinning methods \cite{Ma2002,Sadleir2002} implement the centerline extraction by iteratively remove most of the ``simple points" except the ones that are the end-points of the generated centerline models. %
By ``simple points", it means that the boundary points consisting the object such that their removals do not destroy the topology of the object.

Distance transformation methods \cite{Niblack1992} find the centerline by searching the local maximum among the minimal distances between the points interior of the shape and the boundary. %
However, they do not ensure the resulting centerlines are connected with each other.

``Prairie fire" methods \cite{Blum1967,Leymarie1992} compute the centerline by determining the intersection of the propagating interfaces with their sources located on the boundary of the shape. %

Voronoi diagram method \cite{Sherbrooke1996,Antiga2003} treats the centerline to be generated as a subset of the Voronoi diagram.
These methods are sensitive to the noises and the regularization of the boundary of the shape.

There are numbers of other methods that are not belong to any category listed above.
Ferchichi and Wang in \cite{Ferchichi2006} reported a clustering-based algorithm for centerline extraction both for 2D and 3D objects.
The algorithm was designed based on the idea of computing the maximal disks/balls determining the centerlines, which is achieved by executing the K-means algorithm iteratively on the object-points and their distance transforms. %
Egger \textit{et al.} \cite{Egger2007} reported a centerline extraction algorithm for the blood vessels using Dijkstra's shortest path algorithm, which was designed for the catheter simulation. %
