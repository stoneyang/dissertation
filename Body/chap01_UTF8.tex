%# -*- coding:utf-8 -*- 
\chapter{绪\;\;\;论}
% \chapter{绪论}
\label{chap1}

\section{研究意义}

从二维图像或图像序列中定位、跟踪和识别其中的运动物体,\cite{haag:optical}
并且对运动物体的运动行为作出语义分析,甚至进一步用自然
\subsection{为什么}
语言进行描述,是计算机视觉研究的根本目标。时变图像
序列的语义理解就是要赋予计算机类似于人一样的观察和理解
\subsubsection{就是这样}
动态场景的视觉能力,通过对图像数据的分析与处理来获取对
动态景物中运动物体行为及其相互关系的高层次语义上的解释,
实现从图像空间的数值描述到概念空间的语义描述的转换,最
终达到像人一样能用自然语言来描述动态的外部世界。这也是
当前计算机视觉领域的一个最具挑战性的科学问题。
\subsubsection{就是这样}
动态场景的视觉能力,通过对图像数据的分析与处理来获取对
动态景物中运动物体行为及其相互关系的高层次语义上的解释,
实现从图像空间的数值描述到概念空间的语义描述的转换,最
终达到像人一样能用自然语言来描述动态的外部世界。这也是
当前计算机视觉领域的一个最具挑战性的科学问题。
\subsubsection{就是这样}
动态场景的视觉能力,通过对图像数据的分析与处理来获取对
动态景物中运动物体行为及其相互关系的高层次语义上的解释,
实现从图像空间的数值描述到概念空间的语义描述的转换,最
终达到像人一样能用自然语言来描述动态的外部世界。这也是
当前计算机视觉领域的一个最具挑战性的科学问题。
\subsection{为什么}
语言进行描述,是计算机视觉研究的根本目标。时变图像
序列的语义理解就是要赋予计算机类似于人一样的观察和理解
\subsubsection{就是这样}
动态场景的视觉能力,通过对图像数据的分析与处理来获取对
动态景物中运动物体行为及其相互关系的高层次语义上的解释,
实现从图像空间的数值描述到概念空间的语义描述的转换,最
终达到像人一样能用自然语言来描述动态的外部世界。这也是
当前计算机视觉领域的一个最具挑战性的科学问题。
