%# -*- coding:utf-8 -*-
\chapter{总结与展望}
\label{chap10}

如果上述基本要求能够在较短时间内顺利实现,则把工作的重心转移的实现系统的高级要求上来。
系统的高级特性包括:血管模型应具备与真实血管相近的弹性和形变特性;所重建的心脏区域
(指冠状动脉和心房与心室)可以虚拟心脏的搏动过程;实现虚拟造影剂,该特性在触觉操作机
构的激发下呈现,应当具有与真实手术过程中,医生通过导管向冠状动脉中注入造影剂时的视觉
效果应当一致;实现虚拟支架-气囊,使其能够像真实情况一样--在气囊内部被气体填充的过程
中,支架的结构发生变化、内径逐渐增大;操作过程记录;训练难度分级与训练评价等。

