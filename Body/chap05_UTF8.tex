%# -*- coding:utf-8 -*- 
\chapter{车辆的视觉跟踪}
\label{chap5}
\section{引言}
对于视频序列来说,运动车辆的视觉跟踪是通过对视频序列的每一帧图像不断应用定位算
法来实现的。但是,仅有定位算法是不够的,在跟踪过程中还必须加入一个滤波器,
这是因为:

1)由于图像中存在噪声、模型误差、以及定位算法的误差,在对视频序列的每一帧图
像进行定位时得到的结果往往含有一定的误差。高层视觉信息的处理(如车辆的行为分析)
是建立在跟踪得到的车辆运动轨迹的基础之上的,这些误差将对后续的高层处理带来一定的
困难,需要利用一个滤波器来对跟踪结果进行平滑。

2) 除了平滑作用之外,滤波器还可以起到预测的作用。它可以根据车辆以前的运动信息
对车辆在当前帧中的姿态进行估计和预测,这个预测值将被作为姿态优化的初始值。加入
预测机制不但可以提高定位的速度,而且可以提高定位的稳定性。这是因为在出现严重遮挡
的情况时,姿态评价函数可能是多峰值的,一个好的初始值对于优化结果的正确与否非常
重要。

一般地,一个跟踪预测滤波器的滤波结果好坏取决于两方面的因素:运动模型和滤波器的结构。

本章中提出了一种新的车辆运动模型。
在这个运动模型中考虑了车辆行驶过程中的运动学特性,从而比以前的运动模型更符
合车辆的真实运动。现有的车辆跟踪系统中大多使用了扩展卡尔曼滤波器或者其各种变
体,但是在车辆运动比较复杂时效果不是很好。本章使用了一种改进的扩展卡尔曼滤波器,
加入了一个新的优化目标。实验结果表明,在跟踪车辆运动时,本文中的方法有一定的优点。

\section{车辆的运动模型}
在用滤波器进行运动跟踪和预测时,需要首先确定某种运动模型来对车辆运动进行建模。对于
利用卡尔曼滤波器或者其各种变体进行跟踪的方法来说,运动模型也就是运动系统的状态方程,
描述车辆各个运动状态变化的规律。很显然,运动模型越准确,对车辆的真实运动描述得越好,
有利于滤波器得到更好的跟踪性能。现有的车辆跟踪系统使用的运动模型大多比较简单,例如
\cite{Koller:1993}\cite{Maybank:1996}中使用了匀速圆周运动作为运动模型,这些运动模
型将车辆运动作为一个质点运动来对待,忽略了车辆前后轮之间的运动关系。这种近似性和简
化有时并不合适,制约了滤波器在车辆运动复杂时的性能。本节将根据对车辆的运动学特性的
分析,得到一个更加合理的运动模型。
