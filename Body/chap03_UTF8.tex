%# -*- coding:utf-8 -*-
\chapter{血管模型的提取}
\label{chap3} \fontsize{12pt}{12pt}\selectfont

\section{引言}

动态图像序列的语义化理解是一个系统层次上的课题,本文以交通路面场景的语义理解为研
究对象,它不仅具有明确的应用背景,而且涉及到计算机视觉和图像处理中的很多基本问题,
其中许多算法和问题具有一般性,可以为计算机视觉别的领域所借鉴。开发一个原型的平台对于
交通路面场景的视觉解释研究是至关重要的,不仅可以成为算法验证的实验平台,也可以为
实际应用和推广做准备。在借鉴英国雷丁大学VIEWS项目组的部分经验的基础上,我们设计了
一个拥有完全自主版权的交通监控原型系统ViStar,制作了交通场景模拟平台,并在
PC-Windows平台上用 Visual C++语言初步实现了整个系统平台。
本章将详细地介绍系统平台的框架,以及其中的设计与实现工作,以便读者在深入了解具体
算法之前对整体框架有一个清晰的认识。

\section{系统概况}

如引言中所述,动态图像是一个系统层次上的课题,涉及到计算机视觉和图像处理中的很多
基本问题,有必要开发一个原型实验平台。它既可以作为研究视觉监控算法的实验平台,也
可以为将来的实用化打下坚实的基础。本文的很多工作都是围绕这个实验平台展开的,包括新算
法研究和算法实现,作者所在的NLPR(模式识别国家重点实验室)视觉监控小组就设计和实
现了这样一个拥有完全自主版权的交通场景目标跟踪和行为理解原型实验平台。
