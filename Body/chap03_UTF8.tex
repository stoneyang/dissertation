%# -*- coding:utf-8 -*-
\chapter{基于CTA的血管系统模型的建立}
\label{chap3} \fontsize{12pt}{12pt}\selectfont

\section{引言}

正如本文开篇所述,心血管疾病是一类对人类健康构成严重危害的疾病。
为了对这类疾病进行精确的诊断和治疗,研究人员已经成功研究了许多成像技术(见第\ref{sec2-2}节)。
血管系统本身特有的复杂的空间结构和多变的形态,决定了血管影像中信息量大、特征复杂。
于是,对血管影像的分割就成了一个通往血管模型的精确可视化的关键步骤。
在此基础上,医生才能对病情进行准确的诊断和治疗。
目前,尽管研究人员开展了大量关于血管分割的研究工作——有的已取得不俗成果,有的仍在进行;但是,血管分割仍然是一项极富挑战性的任务\cite{Lesage2009Review}。
随后的工作是对分割所得信息进行图形处理,从而显示血管系统的几何外形。
我们采用行进立方体(Marching Cubes)方法\cite{Lorensen1987MC}将血管外表面所对应的像素值决定的空间曲面提取出来。
为了后续工作的需要,我们采用XXX方法计算血管模型的中心线;
利用XXX方法精简构成血管模型的多边形的数量;
利用XXX方法对多边形数量精简后的血管模型进行多边形三角化;
利用XXX方法对三角化后的血管模型进行几何外观上的平滑处理。

本章的叙述将按照研究工作的顺序依次展开。
首先,介绍基于CTA的血管系统的分割,描述工作中先后采用的两种不同分割技术的原理与实现,以及各自在算法层面的性能,并简要说明这两种技术各自的优势和劣势。
其次,介绍血管模型的可视化,描述工作中采用的主要技术方法的原理与实现。
再次,展示血管建模的实验结果。
最后,给出这部分工作的阶段性结论。

\section{基于CTA的血管系统的分割}

\subsection{基于区域的图像分割方法}

\subsection{基于水平集的图像分割方法}

%\subsection{实际验证}

\section{血管模型的可视化}

\subsection{血管模型的可视}

\subsubsection{面绘制}

Marching Cubes\cite{Lorensen1987MC}是一种被广泛应用于体数据的等值面的提取,该提取结果以多边形的形式来表示。
然而,对于我们的血管介入手术的仿真来说,利用这种方法提取的血管模型的结果缺少拓扑信息,而受到了限制\cite{Nowinski2001NeuroCath}\cite{Hahn1998GWU}。

\subsubsection{中心线提取}

\subsection{面片处理}

\subsubsection{面片精简}

\subsubsection{面片三角化}

\subsubsection{面片平滑}

%\subsection{实际验证}

\section{实际验证}

\section{本章小结} 