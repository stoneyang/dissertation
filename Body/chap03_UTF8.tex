%# -*- coding:utf-8 -*-
\chapter{基于CTA的血管系统模型的建立}
\label{chap3} \fontsize{12pt}{12pt}\selectfont

\section{引言}

正如本文开篇所述,心血管疾病是一类对人类健康构成严重危害的疾病。
为了对这类疾病进行精确的诊断和治疗,研究人员已经成功研究了许多成像技术(见第\ref{sec2-2}节)。
血管系统本身特有的复杂的空间结构和多变的形态,决定了血管影像的大信息量、复杂特征。
于是,对血管影像的分割就成了一个通往血管模型的精确可视化的关键步骤。
在此基础上,医生才能对病情进行准确的诊断和治疗。
目前,尽管研究人员开展了大量关于血管分割的研究工作——有的已取得不俗成果,有的仍在进行;但是,血管分割仍然是一项极富挑战性的任务\cite{Lesage2009Review}。
没有一种分割方法能够独自解决当前所有影像模态获取的影像的分割问题。
分割完成后就要对分割所得信息进行图形处理,从而显示血管系统的几何外形。
我们采用行进立方体(Marching Cubes)方法\cite{Lorensen1987MC}将血管外表面所对应的像素值决定的空间曲面提取出来。
为了后续工作的需要,我们采用XXX方法计算血管模型的中心线;
利用XXX方法精简构成血管模型的多边形的数量;
利用XXX方法对多边形数量精简后的血管模型进行多边形三角化;
利用XXX方法对三角化后的血管模型进行几何外观上的平滑处理。

本章的叙述将按照研究工作的顺序依次展开。
首先,介绍基于CTA的血管系统的分割,描述工作中先后采用的两种不同分割技术的原理与实现,以及各自在算法层面的性能,并简要说明这两种技术各自的优势和劣势。
其次,介绍血管模型的可视化,描述工作中采用的主要技术方法的原理与实现。
再次,展示血管建模的实验结果。
最后,给出这部分工作的阶段性结论。

\section{基于CTA的血管系统的分割}

血管分割是一项难度较大的任务,其难度在于\cite{Lesage2009Review}:
一是影像采集过程中对信息造成的损害,如:对比度、解析度、噪声等;
二是血管系统本身的复杂结构,空间分布复杂、粗细和弯曲程度不一;
三是血管内部的支架、钙化物、血管瘤、管腔狭窄等能对血管系统的外观和几何结构产生影响;
四是血管系统深植于人体复杂的解剖环境当中,彼邻其他器官。
此外,由于血管影像的上述特点,使得血管影像的数据量很大、血管的图像特征很复杂,这都给图像分割算法的运算性能提出了更高的要求。

血管分割技术被广泛应用于脑、心、肺、肝、肾、以及肢体外周等部位的血管系统的分析与诊断。
获取血管影像的医学影像模态包括磁共振和CT等。
根据分割结果的不同,血管分割可分为血管腔分割、血管外壁分割、以及血管内血栓分割\cite{Lesage2009Review}。
由于血管成像时,对比剂的增强效果只能使血管内腔在影像中得以保留,故血管腔分割相对容易;而血管外壁和管内血栓的分割则相对较难。
在本文中,我们仅讨论与经皮冠状动脉导管介入术相关的血管系统的分割,影像模态为CT,影像中包含了病例人体的躯干,扫描时对血管系统进行了增强。
因为我们仅仿真血管内部的情况,所以只需进行血管腔的分割。

用于血管分割的方法多种多样,根据其基本的提取策略可以将这些方法分为区域生长法、活动围线法、以及基于中心线的方法。

\subsection{区域生长方法}

区域生长法是一类基于区域的图像分割方法,它的运行从最初给定的“种子区域”开始,根据使用者预先定义的“生长准则”把符合要求的像素或者子区域合并成为较大区域。
其中,“种子区域”通常是一个或多个像素(二维情形)或者体素(三维情形)。
“生长准则”则是根据处理任务而制定的,其主要限定特征是像素(或者体素)的灰度值或颜色\cite{Gonzalez2004Matlab}。
区域生长法所得结果就是使用者“感兴趣的图像区域”,使用这种方法进行分割的目的就在于此,这一目的能否达成,依赖于“种子”的选取和“生长准则”的制定。
表征区域生长法区别的主要特征有:“生长准则”、确定邻接像素的邻域连接类型、以及寻找邻接像素的策略\cite{Ibanez2005ITKGuide}。
区域生长法主要用于图像分割、以及图像的预处理等场合。

理想情况下,区域生长法的操作始于种子点的选取。
然后,我们将这些种子点分为$n$组(对应于分割目标中的$n$个物体),$A_1, A_2, \ldots, A_n$。
各分组中的种子点的个数可以为$1$,即该组中仅含有一个种子点,分组总数也可以为$1$,即分割目标只有一个。
接着,区域生长法在此基础上会求出一个对作用图像的曲面细分(Tessellation),这个细分将原图像划分为若干区域。
这些区域都具有这样的性质,即一个区域中,每个连接到的分量都与某一组种子点$A_i$相连。
受此约束,所选的区域都尽可能是“齐次的”\cite{Adams1994SRG}。

算法的每一步都在演进,借以向上述的种子点的集合中的其中一个添加一个像素。
现在,我们考虑该算法运行$m$步后,集合$A_i$的状态。
如果我们用$T$来表示所有与上述区域邻接而尚未被重新分配的像素的状态,那么
\begin{equation}
\label{eq3-1}
T = \left\{x\notin\bigcup_{i=1}^{n}A_i|N(x)\cap\bigcup_{i=1}^{n}A_i\neq\varnothing\right\}
\end{equation}
其中,$N(x)$是像素$x$邻接区域中像素所组成的集合。
下面,我们假定所讨论的算法作用于平面图像上,一个像素与其周围的像素的邻接形式是$8$-邻域。
若$\forall x\in T$,
则以$i(x)\in \left\{1, 2, \ldots, n\right\}$为上述某种子点集合$A$的下标,该集合满足$N(x)\cap\bigcup_{i=1}^{n}A_i\neq\varnothing$。
以$\delta(x)$表示$x$与其邻接区域的不同程度。
该$\delta(x)$可表示为
\begin{equation}
\label{eq3-2}
\delta(x)=\left|g(x)-\underset{y\in A_{i(x)}}{mean}[g(y)]\right|
\end{equation}
其中,$g(x)$是图像在像素$x$处的灰度值。
若$N(x)$与两个或更多的$A_i$集合相交,那么我们就将$i(x)$作为$i$的值,使得$N(x)$与$A_i$相交,且$\delta(x)$最小。
或者,在同样情况下,我们可能需要将$x$归为一个边界像素,并将之添加到集合$B$中,该集合包含了那些已经被发现的边界像素。
标定这些边界像素对满足显示需要或者下文提到的半交互修正流程都很有用。
我们取$z\in T$,使得
\begin{equation}
\label{eq3-3}
\delta(z)=\min\limits_{x\in T}{\left\{\delta(x)\right\}}
\end{equation}
并将$z$添加到$A_i(z)$中。

至此,算法就完成了第$m+1$步。
整个算法将依照上面介绍的过程重复运行,直到所有的像素都被恰当地分配为止。
此过程开始于每个种子点的集合$A_i$互不相交。
式\ref{eq3-2}和式\ref{eq3-3}所给出的定义保证在考虑连接性的约束的前提下,最终的分割结果由尽可能“齐次的”区域组成。

在区域生长法的程序设计中,我们采用了串列顺序列表(Sequential Sorted List, SSL),也就是计算机算法理论中的链表(Linked List)。
在区域生长法的情景下,SSL其实就是分割过程中,原始图像中的像素被读入内存后的地址,这些像素地址根据某种属性进行排列。
当在区域生长法的每一步开始时,我们从SSL的起始处取要考虑的下一个新像素。
当一个像素要被添加至SSL的时候,我们必须根据其自身具有的排序属性来选定这个像素在SSL中的位置。
在我们讨论的这种情况中,SSL根据$\delta$来收储集合$T$中的数据。

实现上述算法(边界标定的情形)的伪代码如下:
{\renewcommand{\baselinestretch}{1}{
\begin{algorithm}[htb]                    %算法的开始
\caption{\textsc{Seeded Region Growing}}  %算法的标题
\label{alg:SRG}                           %给算法一个标签,这样方便在文中对算法的引用
\begin{algorithmic}[1]   %不知[1]是干嘛的?
%\REQUIRE ~~\\            %算法的输入参数:Initialization
%  Set $J=0$; $S_0  = \left\{ \phi  \right\}$; $R(S_0 ) = 0$; $\Omega=\{1,2,\ldots,K\}$;
\ENSURE ~~\\             %算法的迭代:Iteration
%Ensemble of classifiers on the current batch,  $E_n$;
%\WHILE    {$J<M$}
%\STATE $J\leftarrow J+1$;
%    \FORALL {$k\in \Omega$}
%    \STATE  $R_{temp}=0$;  $\Delta R_{J,k} = R(S_{J-1}\cup \{k\})-R(S_{J-1})$;
%         \IF  {$\Delta R_{J,k}>0$}
%            \IF{$R(S_{J-1}\cup \{k\})\geq R_{temp}$}
%            \STATE $R_{temp}\leftarrow R(S_{J-1}\cup \{k\})$; $s_J\leftarrow k$;
%            \ENDIF
%         \ELSE
%            \IF{$1/(1-\Delta R_{J,k}/c_k )<rand(1)$}
%            \STATE $\Omega= \Omega-\{k\}$;
%            \ENDIF
%         \ENDIF
%    \ENDFOR
%\IF {$R_{temp}>0$}
%\STATE $S_J\leftarrow S_{J-1}\cup \{s_J\}$;
%\ELSE
%\STATE $J\leftarrow J-1$; Break;
%\ENDIF
%\ENDWHILE              %算法的返回值

\STATE Label seed points according their initial grouping.
\STATE Put neighbors of seed points (the initial T \ref{eq3-1}) in the SSL.
\WHILE {$SSL\neq \varnothing$}
    \STATE Remove first point $y$ from SSL.
    \STATE Test the neighbors of this point: 
    \IF {all neighbors of $y$ which are already labeled (other than with the boundary label) have the same label}
        \STATE Set $y$ to this label.
        \STATE Update running mean of corresponding region.
        \STATE Add neighbors of $y$ which are neither already set nor already in the SSL to the SSL according to their value of $\delta$.
    \ELSE
        \STATE Flag $y$ with boundary label.
    \ENDIF
\ENDWHILE
%\LASTCON ~~\\          %OUTPUT
%  selected user set $S_J$ and weighted sum rate $R(S_J)$;

\end{algorithmic}
\end{algorithm}
}}

区域生长法充分图像的局部空间信息,它能有效地克服利用其他方法进行分割时,所得结果中有可能存在的图像空间不连续的问题。
然而,作为一种低层的图像分割方法,区域生长法也存在着一些不足。
首先,区域生长法在克服了前述问题的同时,容易造成过度分割(使分割结果中包含了一些使用者“并不感兴趣的区域”);
其次,区域生长法依赖人工交互以获得使处理得以进行的“种子”——这是由其基本特征所决定;
最后,区域生长法对图像中的噪声比较敏感,因此在引入这类方法之前,必需对被处理图像进行降噪处理。
除此之外,由于区域生长法属于串行算法,当欲分割目标较大时,处理速度就会明显变慢。
因此,在诊断场合中,此类方法一般用于对肿瘤和伤疤等目标进行分割。

\subsection{快速步进方法}

\cite{Sethian1996FastMarching} \cite{Osher1988LevelSet}

%\subsection{实际验证}

\section{血管模型的可视化}

\subsection{血管模型的可视}

\subsubsection{面绘制}

Marching Cubes\cite{Lorensen1987MC}是一种被广泛应用于体数据的等值面的提取,该提取结果以多边形的形式来表示。
然而,对于我们的血管介入手术的仿真来说,利用这种方法提取的血管模型的结果缺少拓扑信息,而受到了限制\cite{Nowinski2001NeuroCath}\cite{Hahn1998GWU}。

\subsubsection{中心线提取}

\subsection{面片处理}

\subsubsection{面片精简}

\subsubsection{面片三角化}

\subsubsection{面片平滑}

%\subsection{实际验证}

\section{实际验证}

\section{本章小结} 