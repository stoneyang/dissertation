%# -*- coding:utf-8 -*-
\section{问题声明}
\label{sec1-1}

%\subsection{问题的提出}
%\label{sec1-1-1}

经皮冠状动脉介入术仿真训练系统是一个以实现主要功能的软件系统为核心,辅以为受训者提供操作对象的触觉操作机构而成的一个功能完整的混合系统。
该系统能够为受训者提供充分的真实性,使受训者能够“沉浸”在平台所提供的场景中,按照所学的手术技术进行仿真实践,通过这一过程进一步了解手术流程、获得对手术技术更为形象的认识。

与经皮冠状动脉介入术仿真训练系统有关的关键技术包括解剖环境的创建、虚拟手术工具的创建、触觉接口的设计与制造、以及各部分工作的整合等方面。
本文的研究重点是解剖环境的创建,其中包括以下几个问题:
\begin{itemize}
\item 基于CTA的血管系统几何模型的建立;
\item 基于CTA的心脏可视化模型的建立;
\item 器官组织的力学特性分析与建模;
\item 手术仿真的增强性效果的研究;
\item 手术流程的建模等。
\end{itemize}

血管几何模型需要满足的基本要求是:基本准确地分割并重建出若干不同病例的血管系统模型,且能够近乎真实地反映腹主动脉、冠状动脉等血管的主要脉络的几何拓扑结构。

心脏模型需要满足的基本要求是:尽可能准确地分割并重建出相应病例数据中的心脏模型,与血管几何模型实现准确的叠加。

器官组织的力学特性分析与建模的基本要求是:

手术仿真的增强性效果的研究目标是实现一系列的视觉特效,以表现仿真训练过程中的一些视觉效果(如真实导管术中的C型臂X光投影仪的成像效果)、人体的正常生理反应(如呼吸、心脏搏动等),和手术操作效果(如虚拟造影剂)等,从而改善仿真的真实性。%

手术流程建模的目的是在充分分析经皮冠状动脉介入术的正常完整流程的基础上,利用有限状态机原理将整个流程进行状态化的建模。

\subsection{术语解释}
\label{sec1-1-2}

本节将列出一些概念的非正式定义,本文所述内容将围绕这些概念展开。

% template
% \begin{definition}[]
% \end{definition}

\begin{definition}[血管造影]
血管造影是一种介入检测方法。
在血管造影过程中,放射学医师先将显影剂注入血管里,然后在对感兴趣区域(Region of Interest,ROI)进行X光(含CT)或者磁共振检查。
因为X光或者能穿透与其周围组织密度相近的血管和血液,而穿不透显影剂,通过显影剂在X光下的所显示影像来诊断血管病变的。
\end{definition}

\begin{definition}[经皮介入血管成形术]
\end{definition}

% \textbf{图像分割}
\begin{definition}[图像分割]
是指将一幅图像划分为若干部分,各部分与其所在图像中所包含的真实世界中的物体或区域具有强相关性。
%图像分割是介于底层图像处理与图像分析之间的一类图像处理过程,它是通向图像分析的最重要步骤之一。
图像分割的输入可以是原始图像,也可以是某种经过底层图像处理后的预处理图像;其输出则可能是分割得到的“感兴趣”的目标所对应的图像区域或者该区域的闭合边界 \cite{Sonka2008ImageBook,Wang2008ImageBook}。%
%根据任务目的的不同,图像分割可被划分为全局分割和局部分割。
%前者的输出是一组互不相邻的区域,它们与输入图像中的物体一一对应。输入图像则通常包含均一背景中、对比较强的物体;全局分割处理独立于所处环境,无需使用与物体相关的模型,无需利用与期望得到的分割结果相关的知识。
%后者的输出中,所得区域并不直接与输入图像中的物体相对应。输入图像被分成单独的区域,这些区域在选定的属性(如:亮度、色彩、反射性、纹理等)上具有同质性,需要利用较高层次的信息才能完成分割。若输入图像过于复杂,则可能产生同质性区域的相互重叠,这时,还需要先对局部目标进行增强。
%根据分割时所依据的图像特征的不同,图像分割可被划分为基于全局知识(通常是输入图像的特征的直方图)的分割、基于边缘的分割(分割结果是目标所对应区域的闭合边界)、以及基于区域的分割(分割结果是目标所对应的区域)。
\end{definition}

%\textbf{可视化}
\begin{definition}[可视化]
是指利用计算机图形学、图像处理、计算机视觉等领域的方法,为研究人员提供专业数据的视觉显示、增强、以及对信息的交互方式,从而更好地理解数据的技术过程。
可视化技术目前已经应用与科学研究、工业、商业以及医学等领域 \cite{Hearn2004CGBook}。
\end{definition}

%\textbf{渲染}
\begin{definition}[渲染]
是指计算机图形学中,为了获得物体更为真实的可视效果,在场景中的光照和物体的表面特征方面所做的修饰工作。
光照的修饰主要有光源的色彩和空间位置等,物体的表面特征则主要包括透明度、色彩、纹理等 \cite{Hearn2004CGBook}。
\end{definition}

%\textbf{碰撞检测}
\begin{definition}[碰撞检测]
计算机图形学中,研究场景中某两个(或多个)物体是否相交的问题。
具体包括:待检测物体“是否”、“何时”、以及“何处”发生碰撞。
在具体的实现中,碰撞检测技术利用物理学中的相关知识,并且强调运行效果的实时性。
在实践过程中,对碰撞检测的设计产生影像的因素主要有:对象的表达方式、查询类型的多样化、对象所处环境的模拟参数、系统的性能和健壮性、系统实现与应用的简易性等 \cite{Ericson2005CDBook}。%
\end{definition}

%\textbf{形变模型}
\begin{definition}[形变模型]
计算机图形学中,为表达一类非刚体物体(如:绳索、布料、橡胶物品等),而利用外力与内力的相互作用等物理学相关知识来描述这类物体的物体建模方法。
常用方法是通过一张由相互灵活连接(如:弹簧)的节点所组成的空间网络来模拟非刚性物体 \cite{Hearn2004CGBook}。
\end{definition}

%\textbf{中心线提取}
\begin{definition}[中心线提取]
\end{definition}

%\textbf{状态机}
\begin{definition}[状态机]
\end{definition}

%\textbf{粒子系统}
\begin{definition}[粒子系统]
计算机图形学中,用于描述一个或多个使用非邻接片段的物体的概念。
典型的粒子系统被定义于场景中的特定空间中,根据为其设置的参数(如:粒子的路径、色彩、性状等)随时间而发生随机变化。
粒子系统主要用于模拟类似流体特性的现象,如:流动、波浪、喷溅、膨胀、压缩、爆炸等。
具有这类特性的真实现象有:云朵、烟、雾、降雨、降雪、火焰、烟花、瀑布、喷泉、以及大面积水域(江河、湖泊、海洋等)的表面等 \cite{Hearn2004CGBook}。
\end{definition}