%# -*- coding:utf-8 -*-
\section{本文布局}
\label{sec1-4}

本文所述工作的目的是:研究创建冠状动脉介入导管术过程中所涉及的人体解剖环境的虚拟模型的方法,为仿真系统的解剖环境创建功能的实现提供关键技术支持。
冠状动脉介入导管术过程所处的解剖环境包括冠状动脉与主动脉组成的血管网络、冠状动脉围绕的心脏、躯干等。
在本研究项目中,创建解剖环境就是利用图像处理、计算机图形学、生物力学、流体力学等学科的技术,以真实病例的CT血管造影数据为基础,重建解剖环境中各个器官的几何模型,进而模拟其外观和物理特性,最终为使用仿真系统的用户提供足够真实的视觉感受。

本文开篇介绍本研究项目的基本情况,以及与本研究项目有关的同行的研究进展;接下来的每章叙述一个导管术训练系统的关键技术环节;最后一章总结我们在研究工作中的成果,经验,与教训,并展望本项目未来的一些可能的发展动向。全文分为正文和附录等两部分。
正文共八章:第一章是全文绪论,概括介绍本项目的研究动机、研究背景、以及本文的主要内容,详细描述与本文所述工作相关的典型样机和关键技术在国内外研究情况和发展历史;
第二章是医学背景介绍,比较详细地记述了与本文所述工作相关的医学知识,包括介入心脏病学和医学影像技术等内容;
第三章是血管模型的可视化,介绍了为从CTA中获得血管系统所对应的像素信息所进行的主要工作以及工作成果,详细描述本环节工作所用到的关键技术内容,以及在此基础上获得血管系统的可视化模型所进行的主要工作以及工作成果,深入描述了本环节工作所用到的关键技术内容;
第四章是心脏模型的提取与可视化,介绍了基于CTA的心脏模型的分割与可视化工作和工作成果,描述了本环节工作所用到的关键技术内容;
第五章是组织的物理特性建模,具体介绍了器官组织物理特性的分析与建模过程;
第六章是手术过程中特殊动态效果的实现,描述了为实现诸如虚拟造影剂、可视化模型的X光显示等可视效果而做的工作,展示了工作成果;
第七章是手术流程的状态机模型研究,主要介绍将冠状动脉导管术的流程进行计算机建模的一种方法;
第八章是总结与展望,客观、全面地总结全文内容,概括本研究项目所取得的工作成果,列出在此过程中表现出来的一些教训,最后对本研究工作的未来发展做了一些预测。
附录共两章,附录A给出了正文中的一些原理的数学推导;附录B记录了笔者在本项目的研究过程中对软件工程的一些思考。
文末附有参考文献和本人在学期间发表文章的清单等。

本文所述工作是“经皮介入冠脉导管术仿真系统的设计与研究”的一部分,工作任务涉及血管系统和心脏等解剖结构的场景生成和仿真。
关于该仿真系统中的手术工具的物理建模与仿真,请参考组内同事米韶华的博士学位论文《XXX》以及相关的已发表科技文章。

本文所述工作获得了国家自然科学基金(61225017)和北京市优秀博士学位论文指导教师科技项目(YB20108000103)资助。
在此,笔者谨向国家自然科学基金委员会和北京市教育委员会表示由衷的感谢! 