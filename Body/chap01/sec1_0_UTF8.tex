%# -*- coding:utf-8 -*-
第二次世界大战后,世界政治经济的整体状况趋于稳定。
在各国政府的政策引导下,人民饱享经济长期稳定发展、生活水平稳步提升的硕果,而由此带来的生存环境和生活方式的剧烈变化却导致心血管疾病等慢性疾病成为全世界范围最大的流行病 \cite{Hu2009}。%
心血管疾病的形成与个人日常生活中的不良习惯有关,如:吸烟,缺乏体育运动,不健康的饮食习惯,以及超重和肥胖等 \cite{Go2013}。
从生理上讲,促使心血管疾病形成的因素有很多,但大多与动脉粥样硬化有关。
动脉粥样硬化由血液成分的变化引起,当血液中累积过多的脂肪类物质、胆固醇、细胞代谢产生的废物,钙盐,以及纤维蛋白等成分时,就会逐渐附着于血管内壁,最终形成斑块甚至血栓,这些血管内壁上的异物影响血流通畅,最终导致血管疾病\cite{cvdaha}。
目前,以冠心病为代表的心血管疾病的防控形势严峻,据世界卫生组织2010年的报告 \cite{mho2011}称,心血管疾病已成为全球人口的首要死因。
2008年,约有$1730,0000$人死于心血管疾病,约占当年全球死亡人数的30\%,其中超过80\%的死亡人口来自中等收入国家和低收入国家 \cite{mho2011}。
卫生部的相关报告 \cite{moh2010annual,moh2007annual,moh2004annual}显示,在我国的城市和乡村,冠心病的死亡人数正在逐年攀升。
在美国,心血管疾病的死亡数字虽然在相当长一段时间(1999年~2009年)内呈下降趋势,但是由这种疾病所造成的经济负担依然沉重 \cite{Go2013}。
面对这一现实,提高心血管疾病的诊疗水平成为公共卫生管理者和心血管内科学领域的医学工作者的共同目标。
使人类远离疾病困扰、享受健康生活的天然使命促使医学界不断提出新的诊疗方法和药物配方。
其中,以经皮冠状动脉介入术 \cite{Baim2005}为代表的微创介入手术疗法是目前心血管内科学领域普遍采用的、用于治疗冠心病、动脉狭窄、钙化以及动脉血管瘤等病症的有效方法之一,它具有切口小、创伤轻、痛苦少、恢复快等优点,是广大心血管疾病患者的福音。%

在经皮冠状动脉介入术中,医师首先将导丝或导管(为了叙述方便,下文中除非特别说明,均以“导管”来代替这里的“导丝或导管”)通过皮肤穿刺送入病人的血管内;
然后,操作导管沿血管内腔运行,最终将导管送达病灶位置;
然后,再根据对病灶的病理特征的分析,行球囊扩张术或者球囊支架术;
最后,将导管等介入器材从血管中逐一撤出,对创口进行处理。
由于手术的非直观性,因此从事这类手术的医师必须 \cite{Li2012CUHK}:
\begin{itemize}
\item 深入理解血管的解剖结构以及各种导管的物理特性;
\item 克服手眼协调的困难;
\item 选择合适的导管,以进入特定的血管分支。
\end{itemize}
正因如此,医学界对心血管导管术的训练要求相当严格。
根据美国医学毕业生资格审查委员会(Accreditation Council for Graduate Medical Education,ACGME)的培训指南,要从事诊断性导管术的人,在3年心血管疾病资格基础培训里,至少有4个月的诊断性导管术实习(100例患者),附加4个月的导管术实习(附加100例患者) \cite{Beller2002CardTraining}。%
对心脏介入资格的培训,除了三年的普通心血管疾病培训外,附加12个月专业培训,在此期间至少要操作250例介入手术 \cite{Beller2002CardTraining,Hirshfeld1999CardTraining}。
随着新的器械和手术方法的不断引进,心导管介入术医师需要不断更新自己的知识,这就必须完成培训计划,并需要连续操作一定量的手术 \cite{Baim2005}。
由此可见,心血管导管术的训练对于这项手术技术的成功应用具有重要意义。 