%# -*- coding:utf-8 -*-
\section{解决方案}
\label{sec1-2}

本部分将概述构建血管模型、心脏模型、器官组织的生物力学建模、手术仿真的增强效果、以及手术流程状态化等方面的解决方案。

获得血管模型的基本过程是\cite{Preim2008Review}:
\begin{enumerate}
  \item CT数据集进行图像预处理;
  \item 从经过预处理的CT数据集中分割出属于血管模型的像素;
  \item 利用分割所得像素重建出血管模型。
\end{enumerate}
很明显,这个基本思路与实际应用中的CT成像设备的图像处理部分类似,只不过这里我们并不涉及成像和存储部分。

对CT影像进行预处理,这包括:降噪、阈值、图像恢复、重采样、梯度计算、非线性映射、数学形态学处理等方法。
它们的作用是改善图像质量,突出感兴趣的图像细节,为后续的图像分割步骤提供基础。

对进行过预处理的影像数据进行图像分割,图像分割是图像处理中的一项挑战性很强的任务。
到目前为止,都没有一种通用的方法能够应付多数的分割任务。
图像分割方法有很多种,大致可分为:基于区域的图像分割方法、基于边界的图像分割方法以及混合方法等。
我们的候选方法是区域生长法和Fast Marching方法。

区域生长法的基本思路是将具有相似性质的像素集中起来构成区域,该方法需要先在图像中选取一个种子点,然后依次将种子像素周围的相似像素合并到种子像素所在的区域中。
区域生长法的研究重点一是特征度量和区域生长规则的设计,二是算法的高效性和有效性。
区域生长法的优点是计算简单,特别适合分割体积小或者结构相对简单的结构,如肿瘤、伤疤以及主干血管等。
该方法的缺点是需要人工交互以获得种子点,这样使用者必须在每个需要抽取出的区域中植入一个种子点。
同时,区域生长方法也对噪声敏感,导致抽取出的区域有空洞或者在局部体效应的情况下将原本分开的区域连接起来。

Fast Marching法是一种基于几何形变模型的医学影像分割方法。
几何形变模型的理论基础是曲线演化理论和水平集方法。
几何形变模型的基本思想是将曲线的形状变化用曲线演化理论来描述,即用曲率或向量等几何亮度表示曲线或曲面演化的速度函数,并将速度函数与图像数据关联起来,从而使曲线在对象边缘处停止演化。
由于曲线的演化与参数无关,几何形变模型能自动处理对象拓扑的关系,演化过程中的曲线和曲面只能被隐含表示为一个更高维函数的水平集,因此曲线演化过程采用了水平集方法实现。
而Fast Marching方法跟踪运动曲线或曲面的则是固定曲线或者曲面的演化方向,也就是说,曲线或者曲面只能收缩或者只能扩张。

最后一步是在分割所得的模型像素的基础上,利用不同的绘制方法,在场景中渲染出分割所得到的模型。
绘制方法属于计算机图形学范畴,主要的绘制方法有两类:面绘制(或面渲染)和体绘制(或体渲染)。
本课题中的候选方法是Marching Cubes方法和Ray Casting方法。

Marching Cubes方法的基本原理是:采用隐式的等值面提取方法从体数据中获取等值的信息。
算法需要用户提供一个阈值,也就是所希望提取出来的物质的密度,然后根据体数据的信息,就可以把相应阈值内的数据全部提取出来,并且通过三角网格表示出来。

Ray Casting方法的基本原理是:从投影平面的每一个像素点发射出一条光线,穿过三维体数据场,并按照front-to-back的顺序进行采样点光属性的混合,最终得到二维投影图像。
对于三维体数据场中的采样点,可以用最近邻、三线性或样条函数的方法插值出其光属性。
Ray Casting方法得到的可视化效果比较好,立体感强,并且可以很好地实现一些插值算法和光线的提前终止,但是算法的速度比较慢,目前还达不到实时的绘制目的。 