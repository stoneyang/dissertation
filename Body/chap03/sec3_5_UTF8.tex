%# -*- coding:utf-8 -*-
\section{Conclusions and Future Work}
%The conclusion goes here.

The three dimensional model of the aorta is one of the most important parts of the virtual scenarios of the robotic training systems which is built for the intravascular surgical robot.
In this paper, a robust and semi-automatic vessel segmentation approach based on geodesic active contours method has been proposed.
After reporting the design and considerations of the experiments, the results as well as the analysis were presented.
The lumen model of the aorta was then reconstructed by the marching cubes method using the segmented volumetric data.
The experimental results showed that the approach can successfully deal with the messy part due to the anatomic nature of the adherence of the aorta and the spine; and can accurately gain the vessel model.

% The pipeline in this paper was programmed mostly in C++.
%Some of the imaging filters were implemented based on the Insight Toolkit (ITK) \cite{Ibanez2005ITKGuide}.
The data format of the original CTA series were firstly converted.
Then the converted data was fed into the pipeline and was transferred into two branches: one for the production of featured images, and the other for the production of initial level sets.
The production served as the input of geodesic active contours module.
%Moreover, we also provide series of parameters for this module thus it can start the interface evolution.
The evolution stopped after a specified number of iterations and the binary threshold module was called to converse the pixel intensity.
Then the marching cubes implementation extracted the surfaces of the aorta in the segmented data before rendering.

As a future work, on the one hand the coronary arteries needs to be segmented and visualized in the volumetric data.
On the other hand, a series of post-processing work will be carried out in order to eliminate the artifacts and decimate the number of polygons that consisting the surface model.
The whole vasculature model will be available to the simulation tests of the models of the catheters/guidewires.
%Besides, the improvement of the visualization effect of the vessel model will also be an substantial work.
%During this process, a friendly software interface of the surgical simulator will be implemented.
