%# -*- coding:utf-8 -*-
\chapter{躯干部解剖环境的构建}
\label{chap6}
\section{引言}

目前,计算机视觉经过几十年的发展,特别是20世纪八九十年代的研究,使得这一领域中的一些几何问
题已经获得了比较清楚的解释,底层的很多工作都取得了很大进展。但是,视觉的根本问题是赋予计算
机类似于人一样的观察和理解动态场景的视觉能力,通过对图像数据的分析与处理来获取对动态
景物中运动目标行为及其相互关系的高层次语义解释,实现从图像空间的数值描述到概念空间的语义
描述的转换,最终达到使计算机能像人一样用自然语言来描述动态的外部世界。即使我们
全部解决了重建问题,要想进一步有效地解决“理解”问题还是非常不容易的。在实践中,人们发现三
维信息的重建对于最终图像的语义化理解的帮助并不显著。也正是基于这个原因,最近计算机视觉领域
的一些研究者开始重新转向高层视觉,从对静态图像的处理转向对图像序列的视频理解,从基于重建的
识别转向基于运动的识别和事件行为分析。正象我们已经在第二章\ref{chap2}中提及的那样,行为识
别和理解将成为21世纪计算机视觉研究的重点和热点\cite{Mubarak:2002}。

\section{高层处理的流程与框架}
\label{sec6.2}
我们把从数字化图像序列到场景的自然语言描述的整个过程分成四个阶段,如图\ref{fig6.1},分别为:视觉感觉处理(Perceptual
Process)、概念化处理(Conceptualization)、形式化处理(Symbolic Process )和自然
语言语句生成(Sentence Generation)。在每一个阶段都有不同形式的知识和信息的表达,
