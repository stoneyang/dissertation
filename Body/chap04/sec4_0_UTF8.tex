%# -*- coding:utf-8 -*-

The visualization of the coronary vasculature is of utmost importance in interventional cardiology.
Intravascular surgical robots assist the practitioners to perform the complex procedure while protecting them from the tremendous occupational hazards.
Robotic surgical simulation aims to provide support for the learners in both efficiency and convenience.
The blood vessels especially the coronary arteries with rich details are the key part of the anatomic scenario of the virtual training system.
The variations in diameters and directions make the segmentation of the coronary arteries a difficult work.
In this paper, a robust and semi-automatic approach for the segmentation of the coronary arteries is developed.
The approach is based on the multi-scale tubular enhancement and an improved geodesic active contours model.
The demonstrated approach firstly enhances the tubular objects by computing their ``vesselness".
Next the edge potential maps are calculated based on the enhanced information.
Meanwhile, the initial contours are generated by a modified fast marching method.
Then the actual wave fronts evolution extracts the details of the coronary arteries.
Finally the visualization model is organized based on the segmentation results by the marching cubes method.
This approach has been proved successful for the visualization of the coronary arteries based on the CTA information.

\section{Introduction}
\label{sec4_0}

Coronary artery diseases are one of the key causes of deaths in the modern world. % \cite{WHO2013}.
The fatty blood clot (or plaque) adhering to the inner wall of the tiny vessels can partly or completely block the supply of oxygen and other nutritious substances for the heart muscles.
The diseases may lead to serious or even fatal health problems, such as angina and heart attack \cite{OCallaghan2002}.
%The unhealthy living habits have been proved to be the main reasons that cause the fatal diseases \cite{Go2013}.
Percutaneous coronary intervention (PCI) is the standard clinical solution to the coronary heart diseases.
Comparing with the traditional paradigm of the thoracic surgery, this procedure causes much less incisions and trauma, shorter operation time and post-op observation.
Therefore, this skill lies at the core place in the toolkit of a cardiologist.
Regarding the minimally invasive feature of PCI, the clinicians need to master the insight of the complicated anatomic structures of the coronary arteries and the manipulation of the tools.
Moreover, they must conquer the difficulties of the coordination between eyes and hands during the procedure \cite{Li2012CUHK}.
However, risks arise.
Instances are orthopedic injuries turn out due to the prolonged standing and the heavy weight of the lead protection apron during the PCI procedure \cite{Goldstein2004}, and high incidence rate of brain tumors caused by the long-term ionizing radiation under the fluoroscope \cite{Roguin2012}.

To gain the mastery of this important and complex technique, one must endure sufficient and strict drills under the supervision of the mentor before his/her solo surgery.
Most of the medical institutions and hospitals used to train their residents/interns who aim to be a cardiologist mainly based on human cadavers and living animals \cite{Lunderquist1995}, as well as non-biological models \cite{Mori1998}.
Although these training methods successfully protect the trainees from the common occupational hazards, they also have their flaws:
there is no blood circulation in the human cadavers;
the animal's anatomic structures are different from the human's;
and the preservation of the cadavers and the raising of the animals cost a lot of money and all of them cannot be reused.
The physical models need to be routinely maintained and have limited lifetime, despite they can provide intuitive appearance of the human body.

Computer-aided surgical simulation demonstrates its unique characteristics, such as radiation-free, rich details of the procedure-related anatomy, schedule-free, and ease of maintain.
Numbers of simulation systems were developed in research institutions \cite{Dawson1996DK,Wang1997ICard,Cotin2000ICTS} and technological incorporations \cite{CAEWeb,MenticeWeb,SimbionixWeb}.

Intravascular surgical robots provide the cardiologists brand-new facilities to perform the PCI procedures \cite{NIOBEWeb,HansenWeb,Beyar2006RNS,Smilowitz2012}.
Since the manipulation of the robots are different from the traditional procedure, one needs to learn and practice in a series of lessons.
In this training process, new problems emerge.
First, training on the real robot systems is a huge waste of high-end medical resources.
Second, the robotic training shares the same problems with its traditional counterparts.
Robotic surgical simulation for the da Vinci system have proved successful in solving the problems \cite{Liss2012,Kesavadas2011}.

The aim of this work is to develop an approach to visualize the coronary arteries based on the computed tomography angiography (CTA).
The acquired model will be the critical part of the blood vessel model in the robotic intravascular surgical simulator designed for the robot-assist intravascular surgical system \cite{Ji2011EMBC}.
The segmentation of the coronary vasculature is a difficult and demanding work.
Due to their tiny scale and complex topology, the coronary arteries in CTA are often in relatively low intensities and the complicated details may get lost during the processing.
To address this problem, an approach based on the multi-scale tubular enhancement and an improved geodesic snakes is designed.
For a better segmentation, the vessels are enhanced at first.
The pixels are convolved with a Gaussian kernel to compute the Hessian matrix.
Then we take the eigenvalues of the matrix to calculate the ``vesselness" measure.
Next the speed images are generated by computing gradient and applying nonlinear intensity mapping to the enhanced images.
Simultaneously the initial level sets are generated by the improved fast marching algorithm.
After the above computation complete, the actual fronts propagation starts and the final segmentation results are conversed after the propagation ends.
Then the visualization model of the coronary arteries are extracted.
The experimental results demonstrate that the approach is capable of visualizing the coronary arteries in the CTA.

The rest of this paper is organized as follows.
Section II outlines the precessing work flow and details the techniques introduced in the segmentation tasks.
Section III describes the experiments and presents the results.
The final section concludes the whole work. 