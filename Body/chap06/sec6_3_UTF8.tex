%# -*- coding:utf-8 -*-

\section{Conclusions and Future Work}
%The conclusion goes here.

In the context of simulating the intravascular surgical procedure, the interactive performance between the virtual surgical tools (i.e., catheters or guidewires) and the vasculature is one of the most important benchmarks. %
The amount of the data used to model the surface of the related vasculature is one of the main factors.
In this paper, we developed an approach in order to address this problem.
The basic idea is to eliminate the polygon components consisting the model surface without destroying the geometry of the vasculature.
The data used in the experiments were the patient-specific vessel model generated by series of image processing on the original CT dataset in our previous work.

First of all, the connectivity between the adjacent vertices was validated and the unconnected vertices were removed.
Then the noisy surfaces introduced from the image acquisition and processing were smoothed with the least effects on the model's geometry.
Finally the decimation pass was introduced to reduce the number of the polygon components at the desired reduction rate.
In reducing the polygonal elements, not only did we consider the number of the polygons that had been removed, but also the visualization of the resultant models.
Experimental results illustrated the effectiveness of our approach in reducing the quantities of the polygonal elements in the surface model.

Our future work will be the visualization of other human organs (e.g., heart, skeletons, etc.) appeared in the field of real surgeries on the one hand, and the physical modeling of the virtual human organs on the other. %
