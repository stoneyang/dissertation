%# -*- coding:utf-8 -*-

%Cardiovascular diseases are among the most fatal illnesses in the world.
Percutaneous coronary intervention is the gold standard to coronary diseases in the past decades due to much less trauma and quick recovery.
%On the other hand, for the guidance of the fluoroscope required by the procedure, the clinicians have to suffer high risks of tumors and other diseases.
%Intravascular robotic systems are applied to change this situation.
%Hence the clinicians need to participate full training projects to ensure the flexible manipulation of the systems.
However, due to its minimally invasive traits, clinicians have to defeat the difficulties in eye-hand coordination during the procedure, which also makes it a non-trivial task in the training lab. %
The computer-aided surgical simulation is designed to provide a reliable tool for the early stage of the training process.
In this simulation system, the surface model of the vessels contribute the major part in the virtual anatomic environment.
On the other hand, heavy interactions between the virtual surgical tools and the model surface occur during the training.
In order to achieve acceptable performances, the patient-specific vessel surface model needs further process to adapt to this situation.
We proposed in this paper an approach to optimize the meshes that consist the surface model with its application in consideration.
The connectivity of the surface model is firstly checked.
Next a smooth processing is applied without modifying the geometry of the largest-connected surface.
Then the quantities of the polygons consisting the model surface are eliminated both dramatically and appropriately.
Experimental results illustrated the capability of the proposed approach in optimizing the image-based surface model of human vessels.
The resultant surface model can be used as the validation and experimentation involving the virtual surgical tools.

\section{Introduction}
\label{sec6_0}

%Coronary heart diseases () are among the most fatal illnesses worldwide \cite{WHO2013}.
%The diseases occur when the stenosis or even blockages formed due to the build up of fatty substances on the inner wall of the blood vessels.
Percutaneous coronary intervention (PCI) is the gold standard in fighting the coronary heart diseases, which is one of the most lethal diseases in the modern world \cite{WHO2013}. %
Due to the minimally-invasive traits, this procedure causes smaller incision and much less trauma to the patients than open surgery performed in the past.
%In addition, the hospitalization after the procedure is in turn dramatically shortened.
However, this feature also makes it hard for the learners to get used to it.
Besides, after entering the real catheterization labs, the clinicians have to endure the potential occupational hazards year after year \cite{Smilowitz2012}.
%They typically need strict and long term training before performing the procedure on real patient. %
%What's worse, the practitioners in catheterization labs have to expose themselves under the ionizing radiation from the fluoroscope while examining the morphologies of the patients' vasculature. %

In this context, robot-assisted intravascular system came into play with the aim of changing this situation.
With these robots \cite{Beyar2006RNS,Smilowitz2012}, the work load of the cardiologists are dramatically lightened, and the risks of their health are greatly decreased.
However, as their ancient manually performed counterparts, the technique of steering the robotic arms in surgery is still hard for the novice.

Traditional paradigm of PCI training strictly followed the physical fashion -- performing demonstrative procedure on biological or non-biological models \cite{Lunderquist1995,Mori1998}. %
The former materials, including unclaimed cadavers and live animals, are ethic-disputed.
Moreover, the expenditure on the preservation and feeding is high-rising, and the distinction in anatomy between human and animal volunteer is apparent even for the amateur. %
The latter are rigid and stiff both in touching and looking.

In the case of training of robotic systems, one fact is that the advanced robotic systems can not be applied to the training for the health of the clinicians and the lives of the patients will be at risk. %
In order to solve this problem, better training vehicles are required.
Thanks to the advancement of computer hardware, we could employ the greatly improved computer system as the platform, on which the dedicated computer-aided training simulation can be realized. %

Our aim is to implement a computer-aided simulation training system for the minimally-invasive intravascular robotic system built in our lab \cite{Ji2011EMBC}.
In realizing this training platform, the temporal performance of interaction between the anatomic structures in computing environment (e.g., the vascular system) and the virtual surgical tools are undoubtedly of the primary concern. %
In order to improve this performance, the quantity of the consisting polygons needed to be greatly reduced.

In this paper, we developed an approach to substantially eliminate the total number of the polygonal elements of the vasculature model.
As the most popular primitive for the modern computer graphics application, the polygons can be rendered in any standard commercial computer graphics systems.
The fact is that in surface rendering, the more precise the results are, the more polygons are required.
This makes the rendering of the huge amount of polygons occupy too much computation resources, especially the memories.
In the field of medical visualization and simulation, models rendered using surface techniques require even more polygons to capture various of complex and delicate human anatomic structures. %
The decimation algorithm introduced in our work is an application independent method to eliminate the polygons (especially triangles) by performing local operations \cite{Schroeder1992}. %

The input data is the patient-specific surface model of the human vasculature based on the computed tomography angiography (CTA).
Firstly, the connectivity of the polygons that consisting the surface is validated thoroughly to include the largest connected polygons that is effective in representing the surface. %
Secondly, the uneven surfaces are smoothed in order to reduce the unnecessary effects on the following decimation.
Finally, numbers of the component polygons of the vessel model are eliminated.
The capability of our approach in reducing the amount of polygonal components of the vessel model is proved by the experimental results.

The remainder of this paper is organized as follows.
Section II outlines the work flow and describes the techniques introduced in this work.
Section III describes the experiments and presents the results, ending with a discussion.
The final section concludes the whole work and outlooks the future plans. 