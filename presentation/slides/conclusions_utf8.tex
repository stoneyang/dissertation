%# -*- coding:utf-8 -*-
\section{总结与展望}

\subsection{总结}
\begin{frame}
\begin{enumerate}
\item 人体心血管系统表面模型数据的提取
\begin{itemize}
\item 提出了基于GAC的处理方法,得到主动脉内腔表面模型
\item 提出了基于CURVES的处理方法,得到冠状动脉树状结构模型
\item 通过引入管状物增强技术改进上述方法,得到更完整的冠状动脉树状结构模型
\item 提出了基于GAC的改进方法,得到心脏的粗略表面模型,便于判断冠脉分支
\end{itemize}
\item 面向交互仿真的人体心血管模型的数据可视化处理
\begin{itemize}
\item 提出了精简表面模型的多边形面片的方法,所得模型提高了手术仿真的交互效率
\item 提出了基于Voronoi图的类水平集演进方法的中心线提取方法,得到血管表面模型的中心线
\end{itemize}
\end{enumerate}
\end{frame}

\subsection{展望}
\begin{frame}
\begin{enumerate}
\item 手术仿真可视化效果的增强
\begin{itemize}
\item 真实术中的C型臂X光成像效果
\item 血管模型具备与真实血管相近的形变特性
\item 具有跳动效果的心脏区域模型
\item 虚拟的造影剂注入效果
\item 虚拟支架-气囊结构
\item 虚拟正常生理运动,如呼吸等
\end{itemize}
\item 手术过程的状态化建模
\item 手术仿真训练的教学功能
\begin{itemize}
\item 虚拟训练过程的记录
\item 虚拟训练难度的分级
\item 虚拟训练结果的评价
\end{itemize}
\end{enumerate}
\end{frame}

\begin{frame}

\end{frame}

\begin{frame}

\end{frame} 