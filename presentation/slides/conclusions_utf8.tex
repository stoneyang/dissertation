%# -*- coding:utf-8 -*-
\section{总结与展望}

\subsection{总结}
\begin{frame}
\begin{enumerate}
\item \textbf{人体心血管系统表面模型数据的提取}
\begin{itemize}
\pause \item 提出了基于GAC的处理流程,得到主动脉内腔表面模型
\pause \item 提出了基于CURVES的处理流程,得到冠状动脉树状结构模型
\pause \item 通过引入管状物增强技术改进上述流程,得到更完整的冠状动脉树状结构模型
\pause \item 提出了基于GAC的改进流程,得到心脏的粗略表面模型,便于判断冠脉分支
\end{itemize}
\pause \item \textbf{面向交互仿真的人体心血管模型的数据可视化处理}
\begin{itemize}
\pause \item 提出了精简表面模型的多边形面片的流程,所得模型提高了手术仿真的交互效率
\pause \item 提出了基于Voronoi图的类水平集演进流程的中心线提取流程,得到血管表面模型的中心线
\end{itemize}
\end{enumerate}
\end{frame}

\subsection{展望}
\begin{frame}
\begin{enumerate}
\item \textbf{手术仿真可视化效果的增强}
\begin{itemize}
\item 真实术中的C型臂X光成像效果
\item 血管模型具备与真实血管相近的形变特性
\item 具有跳动效果的心脏区域模型
\item 虚拟的造影剂注入效果
\item 虚拟支架-气囊结构
\item 虚拟正常生理运动,如呼吸等
\end{itemize}
\pause \item \textbf{手术过程的状态化建模}
\pause \item \textbf{手术仿真训练的教学功能}
\begin{itemize}
\item 虚拟训练过程的记录
\item 虚拟训练难度的分级
\item 虚拟训练结果的评价
\end{itemize}
\end{enumerate}
\end{frame}

% \begin{frame}

% \end{frame}

% \begin{frame}

% \end{frame} 