\documentclass{article}

\usepackage[top=10mm,bottom=20mm,left=15mm,right=15mm,vmarginratio=1:1]{geometry}

\begin{document}

\title{\bf \Large Lines for Oral Presentation of Paper $\#420$ on IEEE-EMBC 2014}
\author{Fan Yang}
\date{August 18th, 2014}
\maketitle

All the lines are organized into the following enumeration list, in which each item is paired with the corresponding frame in \texttt{420\_slides.pdf} (except the overlayed frames). %
The presentation is scheduled at $11:45 - 12:00$, August 30th, 2014. 
It is planned to be delivered in about 15 minutes, including 12 minutes for presenting and 3 minutes for Q\&A. 

\begin{enumerate}

% 1st frame
\item Good morning, dear colleagues! \\
      I am proudly introducing our work about an approach to optimizing the blood vessels model. \\
	  And the model is acquired for interactive simulation in computer-based PCI training system. 
      
% 2nd frame
\item Here is the outline of the presentation. \\
      First of all, some medical background and the overview of our whole work are given. \\
      Next, the methods and results are introduced in details. \\
      Finally, the work is concluded.

% 3rd frame
\item The whole work is carried out in the context of designing and implementing the computer-aided simulation system for the intravascular surgical robots in our lab. \\
      Our objective in this work is to optimize the blood vessels model to make it suitable for interactive simulation. \\
      The difficulties we have to defeat are: excessive polygons in the surface -- they occupy too much resources during the simulation.
      
% 4th frame
\item CVDs (cardiovascular diseases) are a kind of severe obstructive malformations occurred in cardiovascular system.\\
      Here are some key facts about CVDs from WHO in 2008.\\
      It is one of the top ten killer diseases in the world, and about $17.3$ million people died from it in 2008.

% 5th frame
\item As a minimally-invasive procedure to open up narrowed or even nearly blocked coronary arteries, PCI (percutaneous coronary intervention) is one of the most critical surgical solution to CVDs.\\%
      The procedure is carried out in three major steps near the lesion. \\
      First, the balloon are delivered along the guidewire to the blockage. \\
      Second, the stent is installed while the balloon is expanded to open up the blockage. \\
      After the stent is well set, the balloon and the guidewire are withdrawn sequentially. \\
      The advantages of this procedure are: small incisions, less trauma, lower rate of complications, and shorter hospitalization.

% 6th frame
\item Frequently radiation exposure and prolonged standing during the surgery might lead to severe health problems for the clinicians. \\
      Thus several surgical robots are designed and built for the PCI procedure, including one developed in our lab.

% 7th frame
\item The minimally-invasive feature of PCI procedure makes it hard to teach and learn, especially on eye-hand coordination. \\
      Both the training of manual and robot-assisted techniques are strictly followed the ancient ``master-apprentice" manner.

% 8th frame
\item The disadvantages of traditional learning are as follows:
      \begin{itemize}
      \item patients are at risks of the novices
      \item frequent exposure under fluoroscopy
      \item economical and ethical issues on live animals
      \item fixed sites and schedules
      \item irreversible
      \item conflicts between clinical and training \textbf{for robotic training}
      \end{itemize}

      The advantages of the innovative computer-aided training
      \begin{itemize}
      \item non-patient involvement
      \item non-exposure of fluoroscopy
      \item relatively economical
      \item no constraints of places and schedules
      \item reversible practice
      \item clinical and training are parallel \textbf{for robotic training}
      \end{itemize}

% 9th frame
\item Many research institutions and commercial companies proposed several prototypes and products.\\
      However, the virtual environment rendered in those systems are not patient-specific, or even not patient-based.\\
      Most of them are not developed for training involving surgical robots.\\
      In developing the surgical simulation system, our aim is to provide virtual anatomic structures based on patient-specific information for the training in both manual and robot-assisted manners. %

% 10th frame
\item The main stages in developing the simulator are organized as this.

% 11th frame
\item The key implemented features are presented here. \\
      The blood vessels model in PCI procedure, the virtual guidewire with physical characteristics, and the mechatronical haptic interface. 
	  
% 12th frame
\item Vasculature model is an important part of the intravascular surgical simulation.\\
      Vasculature model can provide:
      \begin{itemize}
      \item intuitive perspective on anatomy structure
      \item virtual scene in which the virtual devices are manipulated
      \end{itemize}

% 13th frame
\item As mentioned before, the challenges in this work is that excessive polygons occupy too much resources in interactive simulation. \\
      This is caused by the following reasons: 
	  \begin{itemize}
      \item complex geometry of blood vessels
      \item noises during image acquisition
      \item distortion during image segmentation
      \item distortion during surface rendering
      \end{itemize}

% 14th frame
\item \textbf{\emph{Overlayed frames}}\\
      {Main steps}
      \begin{enumerate}
      \item connectivity validation
      \item surface smoothing
      \item normals computing
      \item surface decimation
      \end{enumerate}

% 15th frame
\item The figure on the left illustrates the prototype simulator, and the one on the right presents the sub-branch of the whole blood vessels model.


% 16th frame
\item At first, the connectivity between the vertices in the surface model is validated thoroughly.\\
      The aim is to extract the largest possible connected regions among the surface, ensures that no redundant polygons exist after the validation.

% 17th frame
\item Here is the comparision between the quantities of the polygons before and after the validation. \\
      Judging from this table, we can get that the sub-branch model is already largest connected before the validation.

% 18th frame
\item In the second step, the validated surface needs to be smoothed in order to remove the "facets" or "steps" by-produced during the imaging and visualization. \\
      And then, the normal vectors of the surface are computed to mark the "inside" and the "outside".

% 19th frame
\item From the figure on the left, we can the "facets" and "steps" on the surface before smoothing. \\
      In the other figure, the artifacts are gone after the smoothing step.

% 20th frame
\item Here we presented the effects in different settings.\\
      
% 21st frame
\item In the final step, the polygons in the surface model need to be decimated.\\
      The main scheme is to delete vertices first, and then patch the holes.\\
	  Among the surface model, there are five cases for the spatial relation of the vertices:
	  \begin{itemize}
      \item Common cases
      \begin{itemize}
      \item General simple case
      \item Complex case
      \item Boundary case
      \end{itemize}
      \end{itemize}
      \begin{itemize}
      \item Special cases
      \begin{itemize}
      \item Interior edge case
      \item Corner case
      \end{itemize}
      \end{itemize}

% 22nd frame
\item Here are the five cases mentioned before.

% 23rd frame
\item This is the main scheme in decimating the vertices. \\
      For the general simple case: the number of polygons before reduction is $6$, and the number after reduction is $4$. \\
	  For the boundary case: the number of polygons before reduction is $5$, and number of polygons after reduction is $4$. \\
	  Note that the edge in \emph{red} does not belong to the surface before the computation.

% 24th frame
\item This table illustrates the quantities of the polygons after the decimation under different decimation rate.\\
      Note that the quantities of the polygons in the original sub-branch is: $74,307$.

% 25th frame
\item Here are the resultant models by applying different decimation rates.\\
      Observing these results, the models decimated $90\%$ and $99\%$ contain acceptable amounts of polygons.\\
      However, the visualization of $99\%$ unveiled obvious alterations on the geometry of the model.

% 26th frame
\item Benchmarks in the validation (fps $=$ frames per second)
      \begin{itemize}
      \item \textbf{Initial fps}: fps when the virtual guidewire delivered into the model
      \item \textbf{Average fps}: average of the fps at the five specific locations on the centerlines of the model
      \item \textbf{Success}: the virtual guidewire had been delivered to the given location inside the model
      \end{itemize}

% 27th frame
\item Here are the results about the relation between the reduction rates and the fps during the simulation of the decimated models represented in this work.

% 28th frame
\item Here are the results about the relation between the reduction rates and the fps during the simulation of the whole blood vasculature for the 3-D simulation of the PCI procedure. 

% 29th frame
\item Finally, we summarize this work as follows. \\
      An approach to optimize the surface model of the blood vessels is developed. \\
	  The parameters in the approach are tunned through the experiments. \\
	  The performance of the acquired model is validated in the interactive simulation with the virtual guidewire. \\
	  The validation proved success in terms of fps.

% 30th frame
\item In ths future, the work in reconstruction the anatomy related to PCI procedure will be carried on.\\
      On the other hand, the physical modeling of the model will be surely on schedule.

% 31st frame
\item Here are the references mentioned in this presentation.

% 32nd frame
\item That's all. \\
      Thank you for your attention!

% 33rd frame
\item Any questions?

\end{enumerate}

\end{document} 