
\begin{englishabstract}

Percutaneous coronary intervention is a safe and effective treatment for the coronary artery (CA) diseases.
However, both the indirect sight of the lesion, and the flexible characteristics of the surgical tools, make it difficult to learn. 
Traditional training occurs on both the biological plants (e.g., human cadavers, live animals, etc.) and the non-biological ones (e.g., mannequins, etc.). 
For the formers, the expenditures are high, and the plants can not be used repeatedly, let alone the public controversy. 
Moreover, there are no biolgical activities in the cadavers; and the anatomy of the live animals are different from the human's.
For the latters, though the costs are relatively affordable and the plants can be reused, they lack sufficient vividness. 
In this context, the computer-enhanced, patient-specific simulation for PCI is designed to address these issues. 
The simulation system can provide a training platform that are free from radiation, and sterilization, time schedule and location, and the manipulation can be repeated. 
This work is mainly about the visualization of the anatomic environment in the simulation, the main contributions are as follows: 

We propose a new method based on the classic GAC in order to visualize the lumen of the aorta. 

We propose a new method based on the CURVES system in order to visualize the CA. 

We improve the the previous CURVES-centered method by introducing the tubular enhancement in order to visualize more branches of CA.

To improve the efficiency of the interactive simulation, the surfaces are smoothed, and the quantities of the consisting polygons are decimated without destroying the geometry of the model.

To provide the intuition in identifying the branches of CA, the approximate surface model of the heart was acquired by a new approach based on the GAC. 

To extract the centerlines of the blood vessels, we introduce the method based on the Voronoi diagrams of the surface model. 

To enable the replay and assessment function, a model based on the finite state machine is built to present the whole process of the procedure.

\englishkeywords{Medical Imaging, Medical Visualization, Surgical Simulation, PCI}%
\end{englishabstract}
