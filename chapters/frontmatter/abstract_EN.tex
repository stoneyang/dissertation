
\begin{englishabstract}

Percutaneous coronary intervention (PCI) is a safe and effective treatment for the coronary artery (CA) diseases.
Traditional training occurs on both the biological plants and the non-biological ones. 
For the formers, the expenditures are high, and the plants can not be used repeatedly, let alone the public controversy. 
For the latters, though the costs are relatively affordable and the plants can be reused, they lack sufficient vividness. 
To address these issues, the computer-enhanced simulator is designed, which can provide a vivid virtual environment.
Additionally, it is free from radiation, and sterilization, time schedule and location, and the manipulation can be repeated. 
This work is mainly about the visualization of the anatomy for the simulator.

We propose a new method based on GAC in order to visualize the aorta.
Based on the feature image and the initial level sets, the method detects the edges accurately and these edges are rendered as spatial surfaces. 
To provide the intuition in identifying the branches of CA, the approximate surface of the heart was acquired by an improved approach based on the previous one. 

We propose a new method based on the CURVES system in order to visualize the main branches of the CA. 
Based on this approach, we introduce the tubular enhancement technique to highlight more sub-branches of the CA, which are in much darker intensities and tiny scales.
The acquired network of CA with rich details can enable extended visual effect and more complicated training tasks.

To improve the interactive simulation with the virtual guidewire, the polygons are decimated without apparently destroying the geometry of the model (nearly $99 \%$).
With this approach, $10$fps is achieved in the simulation, which is plausible at present.
To extract the centerlines, we introduce the method based on the Voronoi diagrams. 
This information on the geometry will be benefit for the modeling of the interventional procedure.

\englishkeywords{Medical Imaging, Medical Visualization, Surgical Simulation, PCI}%
\end{englishabstract}
