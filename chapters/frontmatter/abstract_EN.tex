
\begin{englishabstract}

Percutaneous coronary intervention (PCI) is a safe and effective treatment to the coronary artery diseases.
% Compared to the open surgery, the technique demonstrates its advantages, such as small incisions, less traumas, shorter periods of the operation and hospitalization, less risks of complications, etc. 
However, both the clinian's indirect sight of the lesion, and the flexible characteristics of the surgical tools, make it difficult to learn. 
% Besides, the risks of the occupational hazards including brain tumors and orthopedic injuries are very high due to the prolonged ionic exposure and long standing during the surgeries. 
% Robotic-assisted intravascular systems are designed to help the clinicians holding and manipulating the surgical tools during the whole surgery under the ionic exposure.
% However, it is also very hard in the training of the robotic systems. 
Traditional training occurs on both the biological plants (e.g., human cadavers, live animals, etc.) and the non-biological ones (e.g., mannequins, etc.). 
For the formers, the expenditures are high, and the plants can not be used repeatedly, let alone the criticise from the public. 
Moreover, there are no biolgical activities in the cadavers; and the anatomy of the live animals (e.g., porcine, etc.) are different from the human's.
For the latters, though the costs are relatively affordable and the plants can be reused, they lack sufficient vividness. 
In this context, the computer-enhanced, patient-specific simulation system of the PCI technique is designed to solve these issues. 
This work is mainly about the visualization of the anatomic environment in the simulation, the main contributions are as follows: 

We propose a new method based on the classical geodesic active contours (GAC) in order to visualize the lumen of the aorta. 

We propose a new method based on the CURVES system in order to visualize the coronary arteries. 

In order to extract the centerlines of the blood vessels, we introduce the method of the ``level-set like" evolution based on the Voronoi diagrams of the surface model. 

To improve the efficiency of the interaction between the surgical tools and the surface model of the blood vessels, the surfaces are smoothed by the low-pass filtering technique, and the quantities of the consisting polygons are greatly decreased without destroying the geometry of the model.

The approximated surface model of the heart was acquired by a new approach based on the GAC. 
The heart model can provide the users the intuition in identifying the complicated branches of the coronary arteries. 

\englishkeywords{Medical Imaging, Medical Visualization, Surgical Simulation, Percutaneous Coronary Intervention (PCI), Medical Education}%
\end{englishabstract}
