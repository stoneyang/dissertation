
\begin{englishabstract}
In this thesis, we research the key technologies for coronary intervention simulation. 
The focus is on the generation of human anatomical structures related to the pervasively adopted coronary intervention procedures, especially on the generation of cadiac vasculatures and the heart itself. %
There are still unavoidable issues in doing the surgery and training the surgery when using the traditional way. 
During the training process, the trainee have to face the lack of opportunities to practice due to the uncertainty of the incoming patient's illness, the different anatomical structures between human and live animals��the different responses when applied forces when drilling the procedure on animals, and the subjectivity when the seniors evaluating the trainee's quality of training. %
Our research is aimed to enable the medicine institutes improve the reality of training without the need of real patient or live animals.

The focus of this work is on the geometric modeling and physics modeling of the related anatomical structures. In the appropriately parameterized model, one can perform basic manipulations using virtual surgical tools. %
By adopting this system, we can improve the reality of training, thus the validation of trainee's skills can be judged by standards.

The image segmentation technologies based on region and level sets are employed to segment the voxels representing the required anatomical structures from CTA.
Surface rendering technique is used to extract and visualize the voxels. 
Then the data in resulting geometric vessel model is optimized, the visualization of the model is improved. 
Next, we analyze the physical characteristics of the organs, build the physics model, and implement several display effects to miniaturize the surgical manipulations. 
Finally, we develop the state-machine model of the standard workflow of the coronary intervention procedure to provide the log functionality. 
In the end, we design and implement the coronary intervention simulation system -- VitroCathIA.

Coronary intervention simulation training is the promising application of virtual-reality technologies in the field of medical education, especially in the direction of interventional cardiology. %
In this thesis, we study the key technologies for the surgical simulators for the coronary intervention procedures, which involves many disciplines, such as, system engineering, material engineering, robotic engineering, computer science, biomedical engineering, and medicine, etc. %
The main contributions of this thesis include following issues:
\begin{enumerate}
    \item We propose a simple but convenient method for camera calibration in traffic scenes, an improved motion detection with less sensitivity to lighting, an efficient and robust vehicle localization algorithm. %
    \item We propose a modified extended Kalman filter incorporated with a precise kinematics model for visual vehicle tracking. By being combined with an additional orthogonality condition, the filter has less sensitivity to the time varying model of system. %
        Experiments show that the filter has a good performance when the tracked car is in a complex motion.
    \item In this thesis, a framework for semantic interpretation of vehicles and pedestrians' actions is proposed for practical applications in visual traffic surveillance.  
        We introduce a conceptual space to bridge the gap between low-level processing which is quantitative and high-level processing where information is handled by qualitative means. %
    \item From human's \`mental experiences, there are two aspects of abstraction: ``generality'' and ``complexity". We deal with them in two deferent computational stages named ``conceptual process'' and ``symbolic process'' to simplify the modeling and inference for a computational aim. %
    \item We propose a new interval-based model of action and a temporal analyzer to model and recognize the targets' behaviors in traffic scenes. 
        A single object's behaviors and its interactions with other objects can be handled in the same framework. 
        Finally, some of the recognized actions can be selected and translated into natural language descriptions by some simple grammar rules.
    \item We develop a demo platform for further research which can work at speed of 17 frames per second on a computer with PIV 1.7G CPU and Windows operating system. 
        Now, the system can give some simple semantic interpretations of vehicle's behaviors.
\end{enumerate}

In a word, in this thesis, we have made a lot of fruitful attempts and significant progresses on our simulation system.

\englishkeywords{Virtual reality, Surgical simulation, Medical visualization, Medical imaging, Collision detection, Deformable models, Tissue modeling, Percutaneous coronary intervention}%
\end{englishabstract}
