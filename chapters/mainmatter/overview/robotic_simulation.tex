\section{����ѵ�����������˲����ߵ��������ϵͳ}
\label{sec:robotic_simulation}

The invention of medical robots has changed surgical practice as tasks are simplified and procedure time is reduced, while the clinicians are protected from the prolonged radiation exposure and the hand trembling is isolated.
Medical robots have been widely used in various clinical specialties, examples are orthopedics, neurosurgery, etc.
Nowadays, robot-assisted systems are also being developed and validated in catheterization labs in the hope of applying this innovative technology to save more lives.
A number of successful cases are Niobe magnetic navigation system from Stereotaxi Inc. \cite{NIOBEWeb}, the SenseiRobotic Catheter System from Hansen Medical \cite{HansenWeb}, Remote Navigation System (RNS) from Navicath \cite{Beyar2006RNS}, and CorPath 200 from the alliance of Phillips and Corindus Inc \cite{Smilowitz2012}.

Computer-aided surgical simulation emerged and had quickly become an innovative and safe method in the training of many different medical procedures.
Numbers of robotic surgical simulators for the training of the famous da Vinci surgical robot are designed.
For example, dV-Trainer from Mimic \cite{Liss2012}, RoSS from Simulated Surgical Systems \cite{Kesavadas2011}, ProMIS from CAE \cite{Jonsson2011} and SEP robot simulator from SimSurgery \cite{Meijden2010}, etc.
Many research institutions have also devoted in the field of interventional cardiology and various prototypes for vessel intervention were born in 1990s, such as Dawson-Kaufman simulator \cite{Dawson1996DK}, ICard \cite{Wang1997ICard}, ICTS \cite{Cotin2000ICTS}, etc.
ICTS was turned into a commercial product by Mentice \cite{MenticeWeb} after the final work completed.
Some other commercial products were also provided, for instance, CathLabVR manufactured by CAE \cite{CAEWeb}, ANGIO Mentor invented by Simbionix \cite{SimbionixWeb}, etc.
However, to our best knowledge, most of them are designed to train their users how to do PTCA in traditional manners.
