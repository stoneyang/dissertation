
%Cardiovascular diseases are among the most fatal illnesses in the world.
Percutaneous coronary intervention is the gold standard to coronary diseases in the past decades due to much less trauma and quick recovery.
%On the other hand, for the guidance of the fluoroscope required by the procedure, the clinicians have to suffer high risks of tumors and other diseases.
%Intravascular robotic systems are applied to change this situation.
%Hence the clinicians need to participate full training projects to ensure the flexible manipulation of the systems.
However, due to its minimally invasive traits, clinicians have to defeat the difficulties in eye-hand coordination during the procedure, which also makes it a non-trivial task in the training lab. %
The computer-aided surgical simulation is designed to provide a reliable tool for the early stage of the training process.
In this simulation system, the surface model of the vessels contribute the major part in the virtual anatomic environment.
On the other hand, heavy interactions between the virtual surgical tools and the model surface occur during the training.
In order to achieve acceptable performances, the patient-specific vessel surface model needs further process to adapt to this situation.
We proposed in this paper an approach to optimize the meshes that consist the surface model with its application in consideration.
The connectivity of the surface model is firstly checked.
Next a smooth processing is applied without modifying the geometry of the largest-connected surface.
Then the quantities of the polygons consisting the model surface are eliminated both dramatically and appropriately.
Experimental results illustrated the capability of the proposed approach in optimizing the image-based surface model of human vessels.
The resultant surface model can be used as the validation and experimentation involving the virtual surgical tools.