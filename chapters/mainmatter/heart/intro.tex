
\section{����}
\label{sec:heart_intro}

Coronary heart diseases (CHDs), ranking among the top ten list of the non-accidental causes of deaths, are the malformations of the coronary arteries through which the nutrition supply is transported to the heart muscle \cite{WHO2013}. %
The diseases had spread in both developed and developing countries all over the world.
The main reasons that cause CHDs include cigarette smoking, lack of exercises, depressed mood, pressures from work and life, unhealthy diet, and obesity \cite{Go2013}.

As the key procedure in revascularization, percutaneous coronary intervention (PCI) are conducted in the catheterization laboratories of different continents to save hundreds of lives. %
Small incisions and short period of operation make the procedure incur traumas to the patients as slight as possible; whilst make the learning of it more difficult for the novice in the labs. %

In learning this technique, one has to fully understand the vascular anatomy, to choose the proper catheters or guidewires at the right time, and to overcome the difficulties of the eye-hand coordination \cite{Li2012CUHK}. %
The traditional ways of training rely on human cadavers, living animals, physical simulators made of non-biological materials.
However, the expenditure on keeping the cadavers and feeding the animals are extremely high, while the physical simulators can only provide poor intuition for the trainees.
The situation went worse after the invention and application of the surgical robotic systems, i.e., the high-end machinery also require the users to receive proper and sufficient training before the real robot-assisted operation. %
To streamline this process, several computer-aided simulators have been designed and already demonstrated their advantages \cite{Liss2012}.

The ultimate goal is to design and implement a computer-enhanced surgical simulation system for the training of the intravascular surgical robot built in our lab \cite{Ji2011EMBC}.
The whole work is organized into two integral parts: one is the reconstruction of the human anatomy, and the other is the simulation of the surgical tools.
In this paper, we proposed an approach to segment and visualize the heart in the original CT information, aiming to provide sufficient intuition to the users when interacting with the computer-aided surgical simulation system under construction. %
The main idea in this paper is based on our previous work in segmenting the blood vessels \cite{Yang2014ICRA}.
The resultant heart model, enveloped by the coronary arteries \cite{Yang2013ROBIO}, enables the users judging the directions and spatial relationship of the complicated branches of the vasculature more conveniently. %
