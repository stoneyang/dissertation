\section{������}
\label{sec:coronary_conclusions}

The visualization of the coronary arteries tree is the key part in the construction of the virtual scenario for the robotic surgical simulator.
For the complex spatial structures, the visualization work of the coronary arteries is full of challenges.
In this paper, a segmentation method based on the CURVES system has been developed.
The experiments were presented and the results were demonstrated.
The efforts showed that the proposed approach is capable of segmenting and visualizing the complex geometric structures of the coronary arteries tree.

The original CTA images were firstly extracted so that only the heart area with the coronary arteries was included.
Then the extracted images were smoothed without affecting the edges of the vasculature and thresholded in order to trim off the irrelevant contents.
After this, image features calculation and initial contours generation were conducted on the resulting images.
The CURVES evolution began thereafter and the evolution took a specified number of iterations to detect the boundaries.
Then the resulting pixels, extracted by converting their intensities and representing the boundaries of the vasculature, were extracted and organized by the marching cubes method.

Our future plans are the further improvements of the proposed approach such that more details of the tree structure can be extracted.
And the goals also include the optimization of the blood vessel models for the simulation with virtual tools of the robotic surgical simulator.