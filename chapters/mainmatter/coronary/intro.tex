
\section{����}
\label{sec:coronary_intro}

For the complex spatial scale and geometric topology, the coronary arteries are often in low contrast and the details are easily lost during the processing. %
To address this problem, an approach based on the CURVES method \cite{Lorigo2001} is developed in this paper.
At the beginning, series of preprocessing steps are performed on the original images, including region extraction, smoothing, and thresholding.
Next, the image features (or speed images) are generated by first calculating the gradients pixel-wisely and then applying a nonlinear transformation to map the intensities in a specified range to a new interval. %
Meanwhile, the initial level sets are acquired by the fast marching algorithm.
Once the above computation finished, the actual fronts propagation starts until the boundaries of the coronary arteries are met.
Finally the visualization model of the coronary arteries are extracted by the marching cubes algorithm.
The experimental results demonstrate the capability of the approach in extracting the coronary arteries tree in the CTA.

The rest of this paper is organized as follows.
Section II introduces the imaging and visualization process and describes the employed methods and models in their mathematical details.
Section III details the experiments and the results.
The closing section summarizes the whole work.