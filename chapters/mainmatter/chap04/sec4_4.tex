
\section{Conclusions and Future Work}
\label{sec4_4}
%The conclusion goes here.

The three dimensional visualization of the blood vessels plays an important role in the construction of the virtual scenario for the robotic surgical simulator.
Further, the visualization of the coronary arteries is the most critical and difficult work.
Because of the complex spatial topologies, details can be easily lost in the process of segmentation.
In this paper, a vasculature segmentation method based on tubular-enhanced CURVES has been developed.
Then the process of the experiment was presented and the results were demonstrated.
The experimental results showed that the proposed approach is capable of enhancing the tiny dark vessels and visualizing the geometric details of them.

The tubular feature of coronary arteries was enhanced and the speed images were generated.
Meanwhile, the initial contours for the CURVES method were computed by a modified version of fast marching method in another process.
The actual level sets evolution began after the above computation finished and the evolution took a specified number of iterations to detect the coronary arteries.
At the end of the segmentation, the resulting pixels were all conversed in their intensities.
Finally the data representing the surface of the arteries was extracted by the marching cubes method.

Our future plans are the further optimization of the blood vessel models for the simulation with virtual tools of the robotic surgical simulator.
The principle work will be the decimation of the quantity of the triangles consisting the blood vessels and the improvements of the visualization effects of the virtual scenario. 