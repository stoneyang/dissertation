
The visualization of the coronary vasculature is of utmost importance in interventional cardiology.
Intravascular surgical robots assist the practitioners to perform the complex procedure while protecting them from the tremendous occupational hazards.
Robotic surgical simulation aims to provide support for the learners in both efficiency and convenience.
The blood vessels especially the coronary arteries with rich details are the key part of the anatomic scenario of the virtual training system.
The variations in diameters and directions make the segmentation of the coronary arteries a difficult work.
In this paper, a robust and semi-automatic approach for the segmentation of the coronary arteries is developed.
The approach is based on the multi-scale tubular enhancement and an improved geodesic active contours model.
The demonstrated approach firstly enhances the tubular objects by computing their ``vesselness".
Next the edge potential maps are calculated based on the enhanced information.
Meanwhile, the initial contours are generated by a modified fast marching method.
Then the actual wave fronts evolution extracts the details of the coronary arteries.
Finally the visualization model is organized based on the segmentation results by the marching cubes method.
This approach has been proved successful for the visualization of the coronary arteries based on the CTA information.