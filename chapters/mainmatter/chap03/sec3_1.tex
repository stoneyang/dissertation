
\section{����}
\label{sec3_1}

Cardiovascular diseases are one of the main causes of death in the modern world.
%Recent years, they have also become one of the most killing illnesses in mainland China \cite{moh2004annual,moh2007annual,moh2010annual}.
The common causes of CVDs are the obstructive alterations occurred in the cardiovascular system.
Defects in one's life style are proved to play significant roles in the forming of the diseases, ranging from cigarette smoking and lack of physical exercises to the unhealthy diet and overweight and obesity \cite{Go2013}.

PTCA is an important medical skill for the cardiologists and has been proved to be a standard procedure to cure many kinds of CVDs, due to the small incision, short operation time, and quick recovery.
%This technique is an important medical skill for the cardiologists.
For the nature of this technique, the surgeons in this specialty have to gain solid understanding of vascular anatomy, choose the tools (catheters and guidewires) appropriately, and overcome the difficulties of the eye-hand coordination and the lack of sense of tactile \cite{Li2012CUHK}.
The other side of the coin is that during this complex procedure, the surgeons and the catheterization lab staff have to stand beside the bed to manipulate the tools, wearing bulky lead shielding aprons to protect themselves from the fluoroscope.
The harsh working conditions poses inevitable occupational hazards to the surgery practitioners \cite{Klein2009}.
Long hours of standing can raise orthopedic injuries and spinal disc diseases; therefore it leads to reduced working performance \cite{Goldstein2004}.
Moreover, the long-term ionizing radiation may induce an increased risk of brain malignancies \cite{Roguin2012}.
% So it requires sufficient learning and training from novice to master.
% Additionally, there are some precious opportunities for the trainee---to watch the real operations performed by the seniors in the catheterization labs.

The invention of medical robots has changed surgical practice as tasks are simplified and procedure time is reduced, while the clinicians are protected from the prolonged radiation exposure and the hand trembling is isolated.
Medical robots have been widely used in various clinical specialties, examples are orthopedics, neurosurgery, etc.
Nowadays, robot-assisted systems are also being developed and validated in catheterization labs in the hope of applying this innovative technology to save more lives.
A number of successful cases are Niobe magnetic navigation system from Stereotaxi Inc. \cite{NIOBEWeb}, the SenseiRobotic Catheter System from Hansen Medical \cite{HansenWeb}, Remote Navigation System (RNS) from Navicath \cite{Beyar2006RNS}, and CorPath 200 from the alliance of Phillips and Corindus Inc \cite{Smilowitz2012}.

To master the robot-assisted procedure, one must learn how to use the robot in sufficient time before his/her first real surgery.
It means that the training needs certain plants or objects to operate on -- something that provides enough details relating to the anatomy structures.
Traditionally, hospitals and medical institutions used to train their residents in this specialty by performing the procedures on donated or unclaimed cadavers, living animals (especially pigs), and sorts of physical mannequins and their variations made of plastics or some other non-biological materials \cite{Lunderquist1995,Mori1998}.
However, there are still many problems, for instance, cadavers are in scarcity and are dead for a time so that there is no blood flow in their vessels; the anatomy of living animals share only a few similarities to human's and performing biological and medical experiments on them often raise arguments from the society.
What's worse, while both preserving cadavers and raising animals cost too high, they can be used for training purposes just once.
Physical models seem to solve most of the problems: they offer fluid flow in the vessels and mimic anatomic structures, and are relatively affordable and can be reused.
However, they do need to be maintained and they have limited service periods.

Compared to the aforementioned methods, computer-aided surgical simulation demonstrates its advantages:
\begin{itemize}
\item no radiation risks;
\item less loss of details of the related anatomic structures;
\item free from restrictions of the schedule;
\item no extra costs for further use.
\end{itemize}
Computer-aided surgical simulation emerged and had quickly become an innovative and safe method in the training of many different medical procedures.
Numbers of robotic surgical simulators for the training of the famous da Vinci surgical robot are designed.
For example, dV-Trainer from Mimic \cite{Liss2012}, RoSS from Simulated Surgical Systems \cite{Kesavadas2011}, ProMIS from CAE \cite{Jonsson2011} and SEP robot simulator from SimSurgery \cite{Meijden2010}, etc.
Many research institutions have also devoted in the field of interventional cardiology and various prototypes for vessel intervention were born in 1990s, such as Dawson-Kaufman simulator \cite{Dawson1996DK}, ICard \cite{Wang1997ICard}, ICTS \cite{Cotin2000ICTS}, etc.
ICTS was turned into a commercial product by Mentice \cite{MenticeWeb} after the final work completed.
Some other commercial products were also provided, for instance, CathLabVR manufactured by CAE \cite{CAEWeb}, ANGIO Mentor invented by Simbionix \cite{SimbionixWeb}, etc.
However, to our best knowledge, most of them are designed to train their users how to do PTCA in traditional manners.

%Vessel models are the most important virtual scenarios required in this kind of training, thus making vessel segmentation an important step.

In this paper, the aim is to provide the three dimensional aorta model based on real patient's medical data for the robotic training system, which is designed for the intravascular surgical robot developed in our lab \cite{Ji2011EMBC}.
As the first task in constructing the vessel model, segmentation of abdominal aorta from the CTA series is challenging, because the aorta and the bones often adhere together.
This makes the aorta hard to segment cleanly.
To address this problem, a robust and semi-automatic approach is developed.
This approach makes the geodesic active contours method \cite{Caselles1997} as its core function which performs the segmentation on CTA series and produces the 3-D surface model of the abdominal aorta.
The 3-D model provides one of the important parts of the anatomic scenario for the simulation of PTCA procedures.
The experimental results showed that this approach is capable of segmenting the aorta in CTA series.
%The goal of this work is to extract the lumen of the abdominal aorta from CTA series and reconstruct the 3-D surface model of the aorta based on the segmented information.
%The methods and experiences should be easily ported to our further work related to the reconstruction of other organs in the anatomic scenario.

The rest of this paper is organized as follows.
Section II describes of the structure and techniques in the segmentation.
% Then the rest of this section is divided into four parts:
% First, the work flow and the main methods employed to produce the feature images are described.
% Second, the methods for the generation of the initial level set are introduced.
% Third, the core armed with geodesic active contours method is described in mathematical details.
% In the end, the method used for visualizing the segmented vasculature is given.
Section III details the experiments and presents the results.
The final section concludes the whole work. %and predicts the future work.