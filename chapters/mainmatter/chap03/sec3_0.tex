
Percutaneous transluminal coronary angioplasty (PTCA) has been proved to be a standard solution to most cardiovascular diseases (CVDs).
The intravascular surgical robots enable the clinicians to perform the procedure with protection from occupational hazards.
The robotic surgical simulator can provide the trainees a new vehicle to learn this skill much more conveniently and effectively.
%Vessel models are the most important virtual scenarios required in this simulation training.
%Vessel segmentation is an important step for the modeling of the vascular system.
%One of the most critical features of the simulator is the virtual environment that is almost identical to the real anatomic structures.
%The blood vessel model is an indispensable component of the virtual environment.
The blood vessel model is the core component of the virtual environment that is almost identical to the real anatomic structures.
Because of the adherence of the aorta and the bones, the ``clean'' segmentation of the aorta is a challenging task.
In this paper, a robust and semi-automatic approach to segment the abdominal aorta from the computed tomography angiography (CTA) is developed.
The proposed approach provides the geodesic active contours implementation with the edge potential maps acquired by applying nonlinear intensity mapping on the gradient images.
A two-branching pipeline is built to achieve this goal: one branch for the edge potential feature computation, and the other for the generation of the initial level sets.
In acquiring edge potential maps, the magnitude of gradient is computed pixel-wisely on the smoothed images.
Meanwhile, the unrelated image contents are stripped carefully before the unidirectional fronts propagation driven by the fast marching algorithm started.
%For the former branch, the images are smoothed and the magnitude of gradient is computed pixel-wisely and then the potential feature images are produced by applying a nonlinear intensity mapping on the previously acquired gradient maps.
%For the latter branch, the fast marching algorithm is used to generate the initial level sets after the original images have been appropriately thresholded.
Then the geodesic active contours method evolves the contours thus the edge of the aorta is met.
Finally the surface information representing the vessel is extracted by the marching cubes method from the segmentation results.
This approach has been proved successful for the construction of 3-D surface model of the aorta based on the CTA series.