%# -*- coding:utf-8 -*- 
%=========================================================================%
%   LaTeX File for phd thesis of Institute of Automation, CAS
%-------------------------------------------------------------------------%
%   Revised by J. G. Lou (jglou@nlpr.ia.ac.cn)
%   参考文献
%-------------------------------------------------------------------------%
%\begin{thebibliography}{200}
\bibliographystyle{unsrt}
\bibliography{myreference_UTF8}
%\bibitem{Marr:1982} D. Marr, Vision-A Computational Investigation into the Human Representation and Processing of Visual Information. W.H. Freeman, 1982.
%\bibitem{Witkin:1986} A, Witkin and M. Tenenbaum, ``On perceptual organization'', In \textit{From Pixels to Predicates}, A. Pentland (Ed.). pp. 149-169, 1986.
%\bibitem{Buxton:2000} H. Buxton and A. Mukerjee, ``Conceptualizing Images'', \textit{Image and Vision Computing}, Vol. 18, pp. 79, 2000.
%\bibitem{Mubarak:2002} Mubarak Shah, ``Guest Introduction: The Changing Shape of Computer Vision in the Twenty-First Century'', \textit{International Journal of Computer Vision}, Vol. 50(2), pp. 103-110, 2002.
%\bibitem{Shi:1994} J. Shi and C. Tomasi, “Good Features to Track”, Proc.of IEEE Int. Conf. on Computer Vision and Pattern Recognition (CVPR'94), pp. 593 -- 600, 1994.
%\bibitem{Malik:1997} J. Malik and S. Russell, “Traffic Surveillance and Detection Technology Development: New Sensor Technology Final Report”, Research Report UCB-ITS-PRR-97-6, California PATH Program, 1997.
%\bibitem{Peter:1990} K. Peter, Karmann and A. Brandt, ``Moving Object Recognition Using an Adaptive Background Memory”, in V Cappellini, editor, Time-Varying Image Processing and Moving Object Recognition, Elsevier, Amsterdam, The Netherlands, 1990.
%\bibitem{Blake:1997} A. Blake and M. Isard, “Active Contours: The Application of Techniques from Graphics, Vision, Control Theory and Statistics to Visual Tracking of Shapes in Motion'', Springer, 1997.
%\bibitem{Kojima:2000} A. Kojima, M. Izumi, et al. ``Generating Natural Language Description of Human Behavior from Video Images'', In Proceeding of International Conference on Pattern Recognition, pp. 728-731, 2000.
%\bibitem{Kojima:2002} A. Kojima, T. Tamura, ``Natural Language Description of Human Activities from Video ImagesBased on Concept Hierarchy of Actions'', International Journal of Computer Vision 50(2), 171--184, 2002.
%\bibitem{Kollnig:1994} H. Kollnig, H.H. Nagel and M. Otte, ``Association of Motion Verbs with Vehicle Movements Extracted from Dense Optical Flow Fields'', In Proceedings of 3rd European Conference on Computer Vision '94, pp. 338-347, 1994.
%\bibitem{Haag:2000} M. Haag, H. H. Nagel, ``Incremental recognition of traffic situations from video image sequences'', Image and Vision Computing, Vol. 18, pp. 137-153, 2000.
%\bibitem{Herzog:1995} G. Herzog, ``From Visual Input to Verbal Output in the Visual Translator'', In: R.~K. Srihari (ed.), Proc. of the AAAI Fall Symposium on Computational Models for Integrating Language and Vision, MIT, Cambridge, MA, Nov. 10, 1995.
%\bibitem{Gerd:1995} Gerd Herzog, Karl Rohr, “Integrating Vision and Language Towards Automatic Description of Human Movements'', In Proceedings of the 19$^{th}$ Annual German Conference on Artificial Intelligence, KI-95, 1995.
%\bibitem{Artificial:1994} Artificial Intelligence Review Journal, 8, Special Volume on the Integration of Natural Language and Vision Processing, 1994.
%\bibitem{The:1995} The AAAI Fall Symposium on Computational Models for Integrating Language and Vision, Cambridge, MIT, 1995.
%\bibitem{VSAM:1} VSAM Homepage: \href{http://www.cs.cmu.edu/~vsam}{http://www.cs.cmu.edu/$\sim $vsam}.
%\bibitem{Roberts:1} L.G. Roberts, ``Machine Perception of Three-Dimensional Solids'', Optical and Electro-Optical Information Processing, J.P. Tippet et al. Eds. Pp. 159-197, Cambridge, MIT Press.
%\bibitem{Marr:1980} D. Marr and E. Hidreth, ``Theory of edge detection'', In Proc. R. Soc. Lond., vol. B-207, pp. 187-217, 1980.
%\bibitem{Masongde:1998} 马颂德,张正友,计算机视觉――计算理论与算法基础,科学出版社,1998.
%\bibitem{human:1} http://human-brain.org/vision.html
%\bibitem{Cedras:1995} C. Cedras and M. Shah, ``Motion based recognition: A survey'', Image and Vision Computing, Vol. 13(2), pp. 129-155, 1995.
%\bibitem{Fisher:1994} R.B. Fisher, ``Is Computer Vision Still AI'', AI Magazine, Jun. 1994.
%\bibitem{Lowe:1992} D. G. Lowe, ``Robust Model-Based Motion Tracking Through the Integration of Search and Estimation'', International Journal of Computer Vision, Vol. 8, No. 2, pp. 113-122, 1992.
%\bibitem{Stauffer:1999} C. Stauffer and W. E . L. Grimson, ``Adaptive Background Mixture Models for Real-time Tracking'', Proc. of IEEE Int. Conf. of Computer Vision and Pattern Recognition (CVPR'99), Vol.2, pages 246-252, 1999.
%\bibitem{Peter:1991} K. Peter, Karmann and A. Brandt, ``Moving Object Recognition Using an Adaptive Background Memory”, in V Cappellini, editor, Time-Varying Image Processing and Moving Object Recognition, Elsevier, Amsterdam, The Netherlands, 1990.
%\bibitem{Sun:2000} H. Z. Sun, T. Feng and T. N. Tan, ``Robust extraction of moving objects from video sequences'', In Proceedings of the Fourth Asian Conference on Computer Vision, Vol. 2, pp. 961-963 , Jan. 2000.
%\bibitem{Ridder:1994} C. Ridder, O. Munkelt and H. Kirchner, ``Adaptive Background Estimation and Foreground Detection using Kalman-Filtering'', International Conference on Recent Advances in Mechatronics, UNESCO Chair on Mechatronics, pp. 193-199, 1994.
%\bibitem{Huwer:2000} S. Huwer, H. Niemann, ``Adaptive Change Detection for Real-time Surveillance Applications'', Workshop on Visual Surveillance in ECCV2000, 2000.
%\bibitem{Gevers:1998} T. Gevers, A. W. M. Smeulders and H. Stokman, ``Photometric Invariant Region Detection'', In Proceeding of Britsh Machine Vision Conference, pp. 659-669, 1998.
%\bibitem{Wren:1997} C. Wren, A. Azarbayejani, T. Darrell and A. Pentland, ``Pfinder: Real-Time Tracking of the Human Body”, IEEE Transaction on Pattern Analysis and Machine Intelligence, Vol. 19, no. 7, pp. 780-785, 1997.
%\bibitem{Stauder:1997} J. Stauder, ``Segmentation of Moving Objects in Presence of Moving Shadows'', In Proceeding of VLBV 97, Linkoeping, Sweden, July, 1997.
%\bibitem{Cavallaro:2001} A. Cavallaro and T. Ebrahimi, ``Change Detection Based on Color Edges'', In Proceeding of IEEE International Symposium on Circuits and Systems, Sydney (Australia), May 2001.
%\bibitem{Ivanov:1998} Y. Ivanov, A. Bobick and J. Liu, ``Fast Lighting Independent Background Subtraction'', In Proceeding of the IEEE Workshop of Visual Surveillance, pp. 49-55, Bomgay, India, Jan. 1998.
%\bibitem{Eveland:1998} C. Eveland, K. Konolige and R. C. Bolles, ``Background Modeling for Segmentation of Video-Rate Stereo Sequences'', In Proceeding of IEEE Conference on Computer Vision and Pattern Recognition, pp. 266-271, Jun. 1998.
%\bibitem{Gordon:1999} G. Gordon, T. Darrell, M. Harville, J. Woodfill, ``Background estimation and removal based on range and color'', International Conference of Computer Vision and Pattern Recognition, 1999.
%\bibitem{Daniel:2000} Daniel Toth, Til Aach, and Volker Metzler, ``Illumination--Invariant Change Detection”, 4$^{th}$ IEEE Southwest Symposium on Image Analysis and Interpretation, April, 2000.
%\bibitem{Jianguang:2002} Jianguang Lou, Hao Yang, Weiming Hu, Tieniu Tan, ``An Illumination Invariant Change Detection Algorithm'', the 5$^{th}$ Asia Conference on Computer Vision, Australia, 2002.
%\bibitem{Aach:1993} T. Aach, A. Kaup and R. Mester, ``Change Detection in Image Sequences Using Gibbs Random Fields: A Bayesian Approach'', In Proceeding of IEEE International Workshop on Intelligent Signal Processing and Communications Systems, Sendai, Japan, pp. 376-381, Oct. 1993.
%\bibitem{Jens:2000} Jens Rittscher Jien Kato Sebastien Joga and Andrew Blake, ``A Probabilistic Background Model for Tracking”, In Proceeding of European Conference on Computer Vision, 2000.
%\bibitem{Paragios:1999} N. Paragios and G. Tziritas, ``Adaptive Detection and Localization of Moving Objects in Image Sequences,'' Signal Processing: Image Communication, Vol. 14, pp.277--296, 1999.
%\bibitem{Paragios:1996} N. Paragios, P. Perez, G. Tziritas, C. Labit, and P. Bouthemy, “Adaptive detection of moving objects using multiscale techniques”, \textit{IEEE Intern. Conf. on Image Processing}, Switzerland, pp. 525-528, 1996.
%\bibitem{Rasmussen:1} C. Rasmussen and G. D. Hager, ``Probabilistic Data Association Methods for Tracking Complex Visual Objects'', IEEE Transactions on Pattern Analysis and Machine Intelligence, Vol. 23, No.6, June
%2001.
%\bibitem{JPL:1997} JPL, Traffic Surveillance and Detection Technology Development, Sensor Development Final Report, Jet Propulsion Laboratory Publication No. 97-10, 1997.
%\bibitem{Malik:1} J. Malik, S. Russell, J. Weber, T. Huang and D. Koller, ``A Machine Vision Based Surveillance System for Californaia Roads'', PATH project MOU-83 Final Report, University of California.
%\bibitem{Kass:1987} M.Kass, A. Witkin and D. Terzopoulos, ``Snakes: Active Contour Models'', International Journal of Computer Vision, vol.1, pp. 321-331, 1987.
%\bibitem{Andrew:1998} Andrew Blake and Michael Isard, “Active Contours”, Springer-Verlag, 1998.
%\bibitem{Caselles:1997} V. Caselles, R. Kimmel and G. Sapiro, “Geodesic Active Contours” International Journal of Computer Vision, pp. 22:61-79, 1997.
%\bibitem{Kichenassamy:1995} S. Kichenassamy, A. Kumar, P. Olver, A. Tannenbaum and A. Yezzi, ``Gradient Flows and Geometric Active Contour Models”, International Conference of Computer Vision, pp. 810-815, Boston, USA, 1995.
%\bibitem{Malladi:1991} R. Malladi, J. Sethian and B. Vemuri, “Shape Modeling with Front Propagation: A Level Set Approach”, IEEE Conference on Computer Vision and Pattern Recognition, pp. 337-343, 1991.
%\bibitem{Nikos:2002} Nikos Paragios and Rachid Deriche, “Geodesic Active Regions: A new Paradigm to deal with frame partition Problems in Computer vision”, Journal of Visual Communication and Image Representation, pp. 266-280, 2002.
%\bibitem{Nikos:2000} Nikos Paragios and Rachid Deriche, “Geodesic Active Contours and Level Sets for the Detection and Tracking of Moving Objects”, IEEE Transactions on Pattern Analysis and Machine Intelligence, vol. 22, pp. 266-280, 2000.
%\bibitem{Jitendra:1997} Jitendra Malik and Stuart Russell, ``Traffic Surveillance and Detection Technology Development: New Traffic Sensor Technology”, California PATH Research Final Report, UCB-ITS-PRR-97-6, University of California, Berkeley, 1997.
%\bibitem{Koller:1994} D. Koller, J. Weber, T. Huang,J. Malik, G. Ogasawara, B. Rao, and S. Russell, ``Towards Robust Automatic Traffic Scene Analysis in Real-Time'', In Proceeding of IAPR International Conference on Pattern Recognition (ICPR94), pp. 126-131, 1994.
%\bibitem{Malik:2} J. Malik, S. Russell, J. Weber, T. Huang and D. Koller, ``A Machine Vision Based Surveillance System for Californaia Roads'', PATH project MOU-83 Final Report, University of California.
%\bibitem{Fan:1989} Fan, T. J., Medioni, G., and Nevatia, ``Recognizing 3-D Objects Using Surface Descriptions'', IEEE Transaction on Pattern Analysis and Machine Intelligence, Vol.11, pp. 1140-1157, 1989.
%\bibitem{Beymer:1998} D. Beymer, P. Mclauchlan, and J. Malik, ``A Real-time Computer Vision System for Vehicle Tracking and Traffic Surveillance'', Submitted for publication in Transportation Research-C, 1998.
%\bibitem{Malik:1996} J. Malik, S. Rusell, ``Traffic Surveillance and Detection Technolodgy Development (New Traffic Sensor Technology)'', Final Report, University of California, Berkeley, 1996.
%\bibitem{Chachich:1} Chachich, A. Pau, A, Barber, et al., ``Traffic Sensor Using a Color Vision Method, Transportation Sensors and Controls: Collision Avoidance, Traffic Management, and ITS'', SPIE Proceedings, Vol. 2902, pp. 156-165.
%\bibitem{Bernt:2000} Bernt Schiele, ``Vodel-Free Tracking of Cars and People Based on Color Regions'', Proceeding of 1st IEEE International Workshop on Performance Evaluation of Tracking and Surveillance(PETS'2000), Grenoble, France, pp. 61-71, 2000.
%\bibitem{Tan:2000} T. N. Tan and K. D. Baker, Efficient Image Gradient Based Vehicle Localization, IEEE Transaction on Image Processing, vol.9(8), pp. 1343-1356, 2000.
%\bibitem{Tan:1998} T. N. Tan, G. D. Sullivan and K. D. Baker, Model-Based Localization and Recognition of Road Vehicles, International Journal of Computer Vision, vol.27(1), pp. 5-25, 1998.
%\bibitem{Sullivan:1992} G. D. Sullivan, Visual Interpretation of Known Objects in Constrained Scenes, Phil. Trans. Royal Society (B) 337, pp. 361-370, 1992.
%\bibitem{Worrall:1991} A. D. Worrall, R. F. Marslin, G. D. Sullivan, K. D. Baker, Model-Based Tracking, Proc. British Machine Vision Conference (BMVC91), pp. 310-318, 1991.
%\bibitem{Koller:1993} D. Koller, K. Daniilidis and H.-H. Nagel, Model-Based Object Tracking in Monocular Image Sequences of Road Traffic Scenes, International Journal of Computer Vision, vol.10(3), pp. 257-281, 1993.
%\bibitem{Kollnig:1997} H. Kollnig and H-H. Nagel, 3D Pose Estimation by Directly Matching Polyhedral Models to Gray Value Gradients, International Journal of Computer Vision, vol.23(3), pp. 283-302, 1997.
%\bibitem{Haag:1999} M. Haag and H-H. Nagel, Combination of Edge Element and Optical Flow Estimates for 3D-Model-Based Vehicle Tracking in Traffic Image Sequences, International Journal of Computer Vision, vol.35(3), pp. 295-319, 1999.
%\bibitem{Haag:1997} M. Haag, Th. Frank, H. Kollnig and H.-H. Nagel, ``Influence of an Explicitly Modeled 3D Scene on the Tracking of Partially Occluded Vehicles'', Computer Vision and Image Understanding, Vol. 65, No. 2, pp. 206-225, 1997.
%\bibitem{Jacobs:1997} D. W. Jacobs, ``Matching 3-D Models to 2-D Images'', International Journal of Computer Vision, Vol. 21, No. 1, pp. 123-153, 1997.
%\bibitem{Lowe:1993} D. G. Lowe, ``Robust Model-Based Motion Tracking Through the Integration of Search and Estimation'', International Journal of Computer Vision, Vol. 8, No. 2, pp. 113-122, 1992.
%\bibitem{Daucher:1993} N. Daucher, M. Dhome, J. T. Lapreste and G. Rives, ``Modeled Object Pose Estimation and Tracking by Monocular Vision'', In British Machine Vision Conference, BMVC'93, pp. 249-258, UK, 1993.
%\bibitem{Gennery:1992} D. B. Gennery, ``Visual Tracking of Known three-dimensional Objects'', International Journal of Computer Vision, Vol. 7(3), pp.243-270, 1992.
%\bibitem{Brisdon:1987} K. Brisdon, Evaluation and Verification of Model Instances, \textit{Proc. Alvey Vision Conference}, pp. 33-37, 1987.
%\bibitem{Sullivan:1993} G. D. Sullivan, A. D. Worrall, T. N. Tan, et al., Model Based Vision in VIEWS, Report of EP2152: VIEWS, Reading University, UK, 1993.
%\bibitem{Pece:2000} A. E. C. Pece, A. D. Worrall, Tracking without Feature Detection, Pro. 1st IEEE International Workshop on Performance Evaluation of Tracking and Surveillance(PETS'2000), Grenoble, France, pp. 29-37, 2000.
%\bibitem{Pece:1998} A. E. C. Pece, A. D. Worrall, A Statistically-based Newton Method for Pose Refinement. Image and Vision Computing, Vol. 16, pp. 541-544, 1998.
%\bibitem{Lowe:1994} D. G. Lowe, ``Robust Model-Based Motion Tracking Through the Integration of Search and Estimation'', International Journal of Computer Vision, Vol. 8, No. 2, pp. 113-122, 1992.
%\bibitem{Yang:2001} H. Yang, J. G. Lou, H. Z. Sun, W. M. Hu and T. N. Tan, ``Efficient and Robust Vehicle Localization'', In Proceeding of International Conference on Image Processing, pp. 355-358, 2001.
%\bibitem{Kalman:1960} Kalman, R. E. ``A New Approach to Linear Filtering and Prediction Problem'', Journal of Basic Engineering, vol. 82, pp. 35-45, 1960.
%\bibitem{Cox:1994} I. J. Cox and S. L. Hingorani, ``An Efficient Implementation and Evaluation of Reid's Multiple Hypothesis Tracking Algorithm for Visual Tracking'', In Proceeding of International Conference on Pattern Recognition, pp. 437-442, 1994.
%\bibitem{Koller:1995} D. Koller, J. Weber and J. Malik, ``Robust Multiple Car Tracking with Occlusion Reasoning'', In Proceeding of the 3$^{rd}$ European Conference on Computer Vision, pp. 186-196, Stockholm, 1994.
%\bibitem{Haritaoglu:1998} I. Haritaoglu, D. Harwood and L. S. Davis, ``W4: Who? When? Where? What? A Real Time System for Detecting and Tracking People'', In Proceeding of 3$^{rd}$ International Conference on Face and Gesture Recognition, pp. 222-227, Japan, 1998.
%\bibitem{Blake:1995} A. Blake, M. Isanc and D. Reynard, ``Learning to Track the Visual Motion of Contours'', Artificial Intelligence, vol. 78, pp. 101-133, 1995.
%\bibitem{Rosales:1998} R. Rosales and S. Sclaroff, ``Improved Tracking of Multiple Humans with Trajectory Prediction and Occlusion Modeling'', CVPR Workshop on the Interpretation of Visual Motion, 1998.
%\bibitem{Bregler:1997} C. Bregler, ``Learning and Recognition Human Dynamics in Video Sequences'', In Proceeding of International Conference on Computer Vision and Pattern Recognition, pp. 568-574, 1997.
%\bibitem{Robert:1999} T. C. Robert, J. L. Alan and K. Takeo, ``A System for Video Surveillance and Monitoring'', In Proceeding of the American Nuclear Society Eighth International Topical Meeting on Robotics and Remote Systems, Apr. 1999.
%\bibitem{Koller:1996} D. Koller, K. Daniilidis and H. H. Nagel, ``Model-Based Object Tracking in Monocular Image Sequences of Road Traffic Scenes'', Journal of Computer Vision, Vol. 10, No. 3, pp. 257-281, 1993.
%\bibitem{Liu:1998} J. S. Liu and R. Chen, ``Sequential Monte Carlo Methods for Dynamical Systems'', Journal of American Statist Association, vol. 93, pp. 1032-1044, 1998.
%\bibitem{Gordon:1997} N. J. Gordon, ``A Hybrid Bootstrap Filter for Target Tracking in Clutter'', IEEE Transaction on Aerospace and Electronic Systems, vol. 33, no. 1, pp. 353-358, 1997.
%\bibitem{Isard:1996} M. Isard and A. Blake, ``Contour Tracking by Stochastic Propagation of Conditional Density'', In Proceeding of Europe Conference on Computer Vision, pp. 343-356, 1996.
%\bibitem{Maybank:1996} S. J. Maybank, A. D. Worrall and G. D. Sullivan, ``A Filter for Visual Tracking Based on A Stochastic Model for Driver Behaviour'', In Proceeding of the Fourth European Conference on Computer Vision, vol.2, pp. 540-549, UK, Apr. 1996.
%\bibitem{Maybank:1997} S. J. Maybank, A. D. Worrall and G. D. Sullivan, ``Filter for Car Tracking Based on Acceleration and Steering Angle'', In Proceeding of British Machine Vision Conference, pp. 615-624, Sep. 1996.
%\bibitem{John:2001} John K. Tsotsos, ``Motion Understanding: Task-Directed Attention and Representations that Link Perception with Action'', International Journal of Computer Vision, vol. 45(3), pp. 265-280, 2001.
%\bibitem{Badler:1975} N. I. Badler, ``Temporal scene analysis: Conceptual descriptions of objects movements'', Ph.D. Thesis, Department of Computer Science, University of Toronto, 1975.
%\bibitem{Tsotsos:1980} Tsotsos,J., Mylopoulos,J., Covvey,H.D., Zucker,S.W., "A Framework for Visual Motion Understanding", IEEE Pattern Analysis and Machine Intelligence, "Special Issue on Computer Analysis of Time-Varying Imagery", p563 -- 573, Nov. 1980.
%\bibitem{Dance:1993} S. Dance and T. Caelli, ``On the symbolic interpretation of traffic scenes'', In Proceedings of the Asian Conference on Computer Vision (ACCV93), pp 798-801, Osaka, 1993.
%\bibitem{Dance:1994} S. Dance and T. Caelli, “A Symbolic Object Oriented Picture Inter- pretation Network: SOO-PIN”, Advances in Structural and Syntactic Pattern Recognition, Proceedings of the International Workshop, Horst Bunke, pp 530-541, Bern, 1993.
%\bibitem{Dance:1995} S. Dance, T. Caelli, and Z-Q Liu, “An Architecture for A Traffic Scene Interpretation System”, Technical Report No. 94/12, pp. 1-43, Department of Computer Science, University of Melbourne, 1994.
%\bibitem{Liu:2001} Z. Q. Liu, L. T. Bruton, J. C. Bezdek, J. M. Keller, S. Dance, M. R. Bartley and C. Zhang,``Dynamic Image Sequence Analysis Using Fuzzy Measures'', IEEE TRANSACTIONS ON SYSTEMS, MAN, AND CYBERNETICS---PART B: CYBERNETICS, VOL. 31, NO. 4, pp. 557-572, Aug. 2001.
%\bibitem{Dance:1996} S. Dance and Z.-Q. Liu, ``Fuzzy belief and scene interpretation'', in T. Huang and S. Negadharipour, editors, Proceedings of the 1995 IEEE International Conference on Computer Vision, Florida, 1995.
%\bibitem{Dance:1997} S. Dance and Z. Q. Liu, ``High Level Scene Interpretation using Fuzzy Belief'', In Proceeding of ICSC, pp. 258-265, 1995.
%\bibitem{Remagnino:1998} P. Remagnino, T. Tan and K. Baker, “Agent Orientated Annotation in Model Based Visual Surveillance”, In Proceeding of IEEE International Conference on Computer Vision, pp857-862, 1998.
%\bibitem{Huang:1994} T. Huang, D. Koller, J. Malik, G. Ogasawara, B. Rao, S. Russell and J. Weber, ``Automatic Symbolic Traffic Scene Analysis Using Belief Networks'', In Proceeding of 12th National Conferece in AI, pp. 966---972, 1994.
%\bibitem{Sumpter:2000} N. Sumpter and A. Bulpitt, ``Learning Spatio-temporal Patterns for Predicting Object Behaviour'', Image and Vision Computing, vol.18, no. 1, pp. 697-704, 2000.
%\bibitem{Howell:1997} A. J. Howell and H. Buxton, ``Recognising Simple Behaviours using Time-Delay RBF Networks'', Neural Processing Letters (5), pp.97-105, 1997.
%\bibitem{Fernyhough:2000} J. Fernyhough, A.G. Cohn and D.C. Hogg, ``Constructing qualitative event models automatically from video input'', Image and Vision Computing, vol.18(1), pp. 81-103, 2000.
%\bibitem{Wada:2000} T. Wada, T. Matsuyama, ``Multiobject Behavior Recognition by Event Driven Selective Attention Method'', IEEE Trans. On Pattern Analysis and Machine Intelligence, vol.22(8), pp.873-887, 2000.
%\bibitem{Galata:1999} A. Galata, N. Johnson and D. Hogg, ``Learning Structured Behaviour Models Using Variable Length Markov Models”, IEEE International Workshop on Modelling People, Corfu, Greece, Sep. 1999.
%\bibitem{Yamata:1992} J. Yamata, J. Ohya and K. Ishii, ``Recognizing Human Action in Time-Sequential Images Using Hidden Markov Model'', Proceeding of International Conference on Computer Vision and Pattern Recognition, pp. 664-665, 1992.
%\bibitem{Brand:1996} M. Brand, N. Oliver and A. Pentland, “Coupled hidden Markov models for complex action recognition” , Learning and Common Sense Technical Report 407, MIT Media Lab Perceptual Computing,1996.
%\bibitem{Johnson:1998} N. Johnson, “Learning Object Behaviour Models”, Ph.D. thesis, The University of Leeds, Sep. 1998.
%\bibitem{Bobick:1995} A. Bobick and A. Wilson, ``A state-based technique for the summarization and recognition of gesture'', in Proc. of International Conference on Computer Vision, pp. 382-388, Cambridge, 1995.
%\bibitem{Takahashi:1994} K. Takahashi, S. Seki, et al, ``Recognition of dexterous manipulations from time varying images'', in Proc. of IEEE Workshop on Motion of Non-Rigid and Articulated Objects, pp. 23-28, Austin, 1994.
%\bibitem{Bobick:1996} A. F. Bobick and J. Davis, ``Real-time recognition of activity using temporal templates'', in Proc. of IEEE CS Workshop on Applications of Computer Vision, pp. 39-42, 1996.
%\bibitem{Changya:2001} 昌娅, 胡卫明, 谭铁牛. ``交通视觉监控系统中的三维车辆线框模型可视化算法'',工程图学学报增刊,28~33,2001.
%\bibitem{Wang:1991} L. L. Wang, W. H. Tsai, ``Camera Calibration by Vanishing Lines for 3D Computer Vision'', IEEE Trans. on Pattern Recognition and Machine Intelligence, vol. 13(3), pp. 370-376, 1991.
%\bibitem{Huang:1999}J. B. Huang, Z. Chen , J. Y. Lin, ``A Study On The Dual Vanishing Point Property'', Pattern Recognition, vol. 32(12), pp. 2029-2039,1999.
%\bibitem{Caprile:1990} B. Caprile, and V. Torre, ``Using Vanishing Points for Camera Calibration'', International Journal of Computer Vision, vol. 4(2), pp. 127-140,1990.
%\bibitem{Zhang:2000} Z. Zhang, ``A Flexible New Technique for Camera Calibration'', IEEE Transaction on Pattern Analysis and Machine Intelligence, vol. 22(11), pp. 1330-1334,2000.
%\bibitem{Self:1996} S. D. Ma, ``A Self-calibration Technique for Active Vision System'', IEEE Transaction on Robot Automat, vol. 12(1), pp. 114-120,1996.
%\bibitem{Basu:1993} A. Basu, ``Active Calibration: Alternative Strategy and Analysis'', In: Proceeding of IEEE Conference on Computer Vision and Pattern Recognition, New York, 495-500, 1993.
%\bibitem{Sturm:1999} P. F. Sturm, S. J. Maybank, ``On Plane-based Camera Calibration: A General Algorithm, Singularities, Applications'', In: Proceeding of IEEE Conference on Computer Vision and Pattern Recognition, Fort Collins, 432-437, 1999.
%\bibitem{Gevers:1999} T. Gevers, A. W. M. Smeulders and H. Stokman, ``Photometric Invariant Region Detection'', In Proceeding of British Machine Vision Conference (BMVC), pp. 659-669, 1998.
%\bibitem{Liuqf:2002} Q. F. Liu, J. G. Lou, Weiming Hu, Tieniu Tan, "Comparison of model-based pose evaluation algorithm in traffic scenes", the 2nd International Conference on Image and Graphics, Hefei, 2002.
%\bibitem{Liuqf:2003} Q. F. Liu, J. G. Lou, T. N. Tan, W. M. Hu, "POSE EVALUATION BASED ON BAYESIAN CLASSIFICATION ERROR", submitted to ICIP2003.
%\bibitem{Lihezhou:1985} 邹理和、吴兆熊,数字信号处理,国防工业出版社,1985年.
%\bibitem{Zhou:1991} D. H. Zhou, Y. G. Xi and Z. J. Zhang, A Nonlinear Adaptive Fault Detection Filter, International Journal of System Science, vol. 22 (12), pp. 2563-2571, 1991.
%\bibitem{Bobick:1997} A. F. Bobick, "Movement, Activity, and Action: The Role of Knowledge in the Perception of Motion", The Royal Society Workshop on Knowledge-based vision in mam and Machine, London, Feb. 1997.
%\bibitem{Nagel:1988} H. H. Nagel, "From image sequences towards conceptual descriptions", Image and Vision Computing, Vol.16, pp. 73-111, 1997.
%\bibitem{Neumann:1989} B. Neumann, "Natural Language Description of Time-Varying Scenes", In: D. L. Waltz(ed.), Semantic Structures: Advances in Natural Language Processing, pp. 167-207, Hillsdale, NJ: Lawrence Erlbaum, 1989.
%\bibitem{Gardenfors:1991} P. G\"{a}rdenfors, "Conceptual sapces as a framework for cognitive semantics", Proceedings of the Second International Colloquium on Cognitive Science, 1991.
%\bibitem{Gardenfors:1997} P. G\"{a}rdenfors, "Meanings as Conceptual Structures", In Mindscapes: Philosophy, Science, and the Mind, ed. by M. Carrier, Pittsburgh University Press, Pittsburgh, pp. 61-86, 1997.
%\bibitem{Aristotle:0} Aristotle, "On Intepretation", translated by E. M. Edghill, http://www.non-contradiction.com.
%\bibitem{Rohini:1995} Rohini K. Srihari, "Computational models for integrating liguistic and visual information: A survey", Artificial Intelligence Review, vol. 8(5), pp. 349-369, 1995.
%\bibitem{Kosslyn:1990} S. M. Kosslyn, "Mental Imagery", In Daniel N. Osherson et al., editor, \textit{Visual Cognitive and Action}, pp. 73-97, MIT Press, Cambridge Mass, 1990.
%\bibitem{Chella:1997} A. Chella, M. Frixione and S. Gaglio, "A Cognitive Arichitecture for Artificial Vision", Artificial Intelligence, vol. 89, pp. 73-111, 1997.
%\bibitem{Gardenfors:1993} P. G\"{a}rdenfors, "Induction and the evolution of conceptual spaces", in Charles Peirce and the Philosophy of Science: Papers from the Harvard Sesquicentennial Conference, ed. by E.C. Moore, University of Alabama Press, pp. 72-88, 1993.
%\bibitem{Fraile:1998} R. Fraile and S. J. Maybank, "Vehile Trajectory Approximation and Classification", In Proceedings of British Machine Vision Conference, 1998.
%\bibitem{Johnson:1996} N. Johnson and D. C. Hogg, "Learning the Distribution of Object Trajectories for Event Recognition", Image and Vision Computing, vol. 14, pp. 609-615, 1996.
%\bibitem{Stauffer:2000} C. Stauffer and E. Grimson, "Learning Patterns of Activity Using Real-Time Tracking", IEEE Transaction on Pattern Analysis and Machine Intelligence, vol.22, no.8, pp. 747-757, 2000.
%\bibitem{Johnson:1987} M. Johnson, "The Body in the Mind: The Bodily Basis of Reason and Imagination", Chicago: The University of Chicago Press, 1987.
%\bibitem{Stalnaker:1981} R. Stalnaker, "Antiessentialism", Midwest Studies of Philosophy 4: 343-355, 1981.
%\bibitem{Gardenfors:1991b} P. G\"{a}rdenfors, "Frameworks for properties: Possible worlds vs. conceptual spaces", Language, Knowledge and Intentionality, Acta Philosophica Fennica, vol. 49, pp. 383-407, 1991.
%\bibitem{Thibadeau:1994} R. H. Thibadeau, "Artificial Perception of Actions", Gognitive Science, Feb. 1994. http://www.ri.cmu.edu/pubs/pub\_2960.html
%\bibitem{Allen:1983} J. F. Allen, "Maintaining Knowledge about Temporal Intervals", Communications of ACM, Vol. 26, No. 11, pp. 832-843, 1983.
%\bibitem{Allen:1994} J. F. Allen and G. Ferguson, "Actions and Events in Interval Temporal Logic", Journal of Logic and Computation, Vol.4(5), pp. 531-579, 1994.
%\bibitem{Yamato:1992} J. Yamato, J. Ohya, and K. Ishii. "Recognizing Human Action in Time-Sequential Images Using Hidden Markov Model", In Proceedings of International Conference on Computer Vision and Pattern Recognition, pp. 379-385. 1992.
%\bibitem{Wilson:1995} A. Wilson and A. F. Bobick. "Learning Visual Behavior for Gesture Analysis", Proc. of the IEEE-PAMI International Symposium on Computer Vision, Coral Gables, Florida, pp. 229-234. November. 1995.
%\bibitem{Stein:1994} L. A. Stein and L. Morgenstern. "Motivated Action Theory: a Formal Theory of Causal Reasoning", Artificial Intelligence, vol. 71, pp. 1-42. 1994.
%\bibitem{Leith:1997}  M. F. Leith and R. J. Cunningham, "Representing and Reasoning with Events from Natural Language", in Gabbay et al. (eds) Qualitative and Quantitative Practical Reasoning (EQUARU/FAPR 97, Bad Honnef), Springer LNAI 1244, 1997.
%\bibitem{Xiaoling:1994} 左孝凌,李为鑑,刘永才,“离散数学”,上海科学技术文献出版社,1982年第一版,1994第17次印刷.
%\bibitem{Wenxiu:1996} 张文修,梁  怡,“不确定性推理原理”,西安交通大学出版社,1996.
%\bibitem{Talavera:2001} Luis  Talavera, Javier  Béjar, "Generality-Based Conceptual Clustering with Probabilistic Concepts", IEEE Transaction on Pattern Analysis and Machine Intelligence, Vol. 23, No. 2, pp. 196-206, 2001.
%\bibitem{Kogut:2002} G. Kogut, M. Trivedi, "A Wide Area Tracking System for Vision Sensor Networks," In the 9th World Congress on Intelligent Transport Systems, Chicalgo, Oct. 2002.
%\bibitem{Lee:1995} J. W. Lee, M. S. Kim and I. S. Kweon, "A Kalman Filter Based Visual Tracking Algorithm for An Object Moving in 3D", Proc. of the Int'l Conf. on Intelligent Robots and Systems, Aug. 1995.
%\bibitem{Costa:2000}M.S. Costa and L.G. Shapiro, "3D Object Recognition and Pose With Relational Indexing", Computer Vision and Image Understanding, vol. 79, no. 3, pp. 364-407, 2000.
%\end{thebibliography}
