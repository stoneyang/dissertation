%# -*- coding:utf-8 -*- 
%=========================================================================%
%   LaTeX File for phd thesis of Institute of Automation, CAS
%-------------------------------------------------------------------------%
%   张兆翔重新修改了相关格式,以满足最新要求
%-------------------------------------------------------------------------%
%   Revised by J. G. Lou (jglou@nlpr.ia.ac.cn)
%-------------------------------------------------------------------------%

%=========================================================================%
%                          主文档 格式定义
%=========================================================================%

%===================== 重定义字体、字号命令 =============================%
% 注意win2000,没有 simsun, 最好到网上找一个。一些字体是office2000带的
\newcommand{\song}{\CJKfamily{song}}    % 宋体   (Windows自带simsun.ttf)
\newcommand{\fs}{\CJKfamily{fs}}        % 仿宋体 (Windows自带simfs.ttf)
\newcommand{\kai}{\CJKfamily{kai}}      % 楷体   (Windows自带simkai.ttf)
\newcommand{\hei}{\CJKfamily{hei}}      % 黑体   (Windows自带simhei.ttf)
\newcommand{\li}{\CJKfamily{li}}        % 隶书   (Windows自带simli.ttf)
\newcommand{\you}{\CJKfamily{you}}      % 幼圆   (Windows自带simyou.ttf)
\newcommand{\chuhao}{\fontsize{42pt}{\baselineskip}\selectfont}     % 字号设置
\newcommand{\xiaochuhao}{\fontsize{36pt}{\baselineskip}\selectfont} % 字号设置
\newcommand{\yihao}{\fontsize{28pt}{\baselineskip}\selectfont}      % 字号设置
\newcommand{\erhao}{\fontsize{21pt}{\baselineskip}\selectfont}      % 字号设置
\newcommand{\xiaoerhao}{\fontsize{18pt}{\baselineskip}\selectfont}  % 字号设置
\newcommand{\sanhao}{\fontsize{15.75pt}{\baselineskip}\selectfont}  % 字号设置
\newcommand{\sihao}{\fontsize{14pt}{\baselineskip}\selectfont}      % 字号设置
\newcommand{\xiaosihao}{\fontsize{12pt}{\baselineskip}\selectfont}  % 字号设置
\newcommand{\wuhao}{\fontsize{10.5pt}{\baselineskip}\selectfont}    % 字号设置
\newcommand{\xiaowuhao}{\fontsize{9pt}{\baselineskip}\selectfont}   % 字号设置
\newcommand{\liuhao}{\fontsize{7.875pt}{\baselineskip}\selectfont}  % 字号设置
\newcommand{\qihao}{\fontsize{5.25pt}{\baselineskip}\selectfont}    % 字号设置

%===================================================================%
%                         各种距离与缩进
%===================================================================%

%-------------------- 用于中文段落缩进 和正文版式 ------------------%
\CJKcaption{GB_aloft_UTF8}
\setlength{\parindent}{2em}                 % 首行两个汉字的缩进量
\setlength{\parskip}{3pt plus1pt minus1pt}  % 段落之间的竖直距离
\renewcommand{\baselinestretch}{1.2}        % 定义行距

%------------------------- 列表与图表距离设置 -----------------------%
\setlength{\topsep}{3pt plus1pt minus2pt}           % 第一个item和前面版落间的距离
\setlength{\partopsep}{3pt plus1pt minus2pt}        % 当在一个新页开始时加到
                                                    % \topsep的额外空间
\setlength{\itemsep}{3pt plus1pt minus2pt}          % 连续items之间的距离.
\setlength{\floatsep}{10pt plus 3pt minus 2pt}      % 图形之间或图形与正文之间的距离
\setlength{\abovecaptionskip}{2pt plus1pt minus1pt} % 图形中的图与标题之间的距离
\setlength{\belowcaptionskip}{3pt plus1pt minus2pt} % 表格中的表与标题之间的距离

%下面这组命令使浮动对象的缺省值稍微宽松一点,从而防止幅度
%对象占据过多的文本页面,也可以防止在很大空白的浮动页上放置
%很小的图形。
\renewcommand{\textfraction}{0.15}
\renewcommand{\topfraction}{0.85}
\renewcommand{\bottomfraction}{0.65}
\renewcommand{\floatpagefraction}{0.60}

%---------------------------- 数学公式设置 ------------------------------%
\setlength{\abovedisplayskip}{2pt plus1pt minus1pt}     %公式前的距离
\setlength{\belowdisplayskip}{2pt plus1pt minus1pt}     %公式后面的距离
\setlength{\arraycolsep}{2pt}   %在一个array中列之间的空白长度, 因为原来的太宽了

\allowdisplaybreaks[4]  % \eqnarray如果很长,影响分栏、换行和分页
                        %(整块挪动,造成页面空白),可以设置成为自动调整模式

%===================================================================%
%                         各种标题样式
%===================================================================%
%======================= 标题名称中文化 ============================%
\renewcommand\contentsname{目\ 录}
\renewcommand\listfigurename{插图目录}
\renewcommand\listtablename{表格目录}
%\renewcommand\abstractname{摘\ 要} %err undefined
%\renewcommand\refname{参考文献}         %article类型
\renewcommand\bibname{参\ 考\ 文\ 献}    %book类型
\renewcommand\indexname{索\ 引}
\renewcommand\figurename{图}
\renewcommand\tablename{表}
\renewcommand\partname{部分}


%======================= 定制章节的标题样式 =============================%
\setcounter{secnumdepth}{3}
%---------------------- 定义章节的编号格式 --------------------------%
%\renewcommand{\thesection}{\CJKnumber{\arabic{section}}、} % 定义 一、。九、十
\renewcommand{\thesection}{\arabic{chapter}.\arabic{section}}
\renewcommand{\thesubsection}{\arabic{chapter}.\arabic{section}.\arabic{subsection}}
\renewcommand{\thesubsubsection}{\arabic{chapter}.\arabic{section}.\arabic{subsection}.\arabic{subsubsection}}
\let\oldtitle=\title
\def\title#1{\oldtitle{\cnbf{#1}}}
\renewcommand{\CJKglue}{\hskip 0pt plus 0.08\baselineskip}

%----------------------- 定义章节标题格式 ----------------------------%
\titleformat{\chapter}[hang]{\fontsize{15pt}{15pt}\selectfont\filcenter\CJKfamily{hei}}
    {\fontsize{15pt}{15pt}\selectfont{\chaptertitlename}}{20pt}{\fontsize{15pt}{15pt}\selectfont}
\titlespacing{\chapter}{0pt}{-3ex  plus .1ex minus .2ex}{2.5ex plus .1ex minus .2ex}

%\titleformat{\section}[hang]{\CJKfamily{hei}\Large \centering} %标题居中
\titleformat{\section}[hang]{\CJKfamily{hei}\fontsize{14pt}{14pt}\selectfont}
    {\fontsize{14pt}{17pt}\selectfont \thesection}{1em}{}{}
\titlespacing{\section}
    {0pt}{1.5ex plus .1ex minus .2ex}{\wordsep}

\titleformat{\subsection}[hang]{\CJKfamily{hei}\fontsize{13pt}{13pt}\selectfont}
    { \fontsize{13pt}{13pt}\selectfont\thesubsection}{1em}{}{}
\titlespacing{\subsection}%
    {0pt}{1.5ex plus .1ex minus .2ex}{\wordsep}

\titleformat{\subsubsection}[hang]{\CJKfamily{hei}\fontsize{12pt}{12pt}\selectfont}
    {\fontsize{12pt}{12pt}\selectfont\thesubsubsection}{1em}{}{}
\titlespacing{\subsubsection}%
    {0pt}{1.2ex plus .1ex minus .2ex}{\wordsep}

%======================= 定义列表项目格式 ==========================%
\renewcommand\labelenumi{\textcircled{\scriptsize \theenumi}}  %带圈的数字
%\renewcommand\labelenumi{(\theenumi)}
\renewcommand\labelenumii{(\theenumii)}
\renewcommand\labelenumiii{\theenumiii.}
\renewcommand\labelenumiv{\theenumiv.}

%====================== 定制图形和表格标题样式 =====================%
%---------------------- 定制图形和表格标题格式 ---------------------%
\renewcommand{\captionlabeldelim}{}
\renewcommand{\captionlabelfont}{\small \CJKfamily{hei}\bf}
\renewcommand{\captionfont}{\small \CJKfamily{song}\rmfamily}
%% \scriptsize \footnotesize \small \large \Large  %图形标签字体大小

%--------------------- 定义图、表、公式的编号格式 -------------------%
\renewcommand{\thetable}{\arabic{chapter}-\arabic{table}}
\renewcommand{\theequation}{\arabic{chapter}-\arabic{equation}}
\renewcommand{\thefigure}{\arabic{chapter}-\arabic{figure}}


%=========================== 目录设置 ==================================%
\setcounter{tocdepth}{3} \setcounter{secnumdepth}{3}
% 可以用\section[abc]{abcdefg}形式的命令,这样abc就做为缩短标题出现
% 在目录表中和页眉上.另外,还可以利用\addtocounter{secnumdepth}{num}
% 来使得当前章节编号深度增加或减小,num可取正值或负值.

%============================= 页面设置 ================================%
%-------------------- 定义页眉和页脚 使用fancyhdr 宏包 -----------------%
\newcommand{\makeheadrule}{%
    %\makebox[0pt][l]{\rule[.7\baselineskip]{\textwidth}{1.2pt}}
    \rule[.6\baselineskip]{\linewidth}{0.4pt}\vskip-.8\baselineskip}
\makeatletter
\renewcommand{\headrule}{%
    {\if@fancyplain\let\headrulewidth\plainheadrulewidth\fi
     \makeheadrule }} %
\makeatother                                % 定义页眉与正文间隔线

\pagestyle{fancyplain}
\renewcommand{\chaptermark}[1]%
{\markboth{\chaptername \ #1}{}}            % 去掉章节标题中的数字
\fancyhf{} %\fancyfoot[C,C]{\thepage}

\fancyhead[CO]{\CJKfamily{fs}\leftmark}          % 在book文件类别下,
\fancyhead[CE]{\CJKfamily{fs}\rightmark}         % \leftmark自动存录各章之章名,

\fancyfoot[C]%                                  % [RE][LO]
{\CJKfamily{hei} -\;\thepage\;-}

%=== 配合前面的ntheorem宏包产生各种定理结构,重定义一些正文相关标题 ===%
\theoremstyle{plain}
\theoremheaderfont{\normalfont\rmfamily\CJKfamily{hei}}
\theorembodyfont{\normalfont\rm\CJKfamily{song}} \theoremindent0em
\theoremseparator{\hspace{1em}} \theoremnumbering{arabic}
%\theoremsymbol{}          %定理结束时自动添加的标志
\newtheorem{definition}{\hspace{2em}定义}[chapter]
%\newtheorem{definition}{\hei 定义}[section] %!!!注意当section为中国数字时,[sction]不可用!
\newtheorem{proposition}{\hspace{2em}命题}[chapter]
\newtheorem{property}{\hspace{2em}性质}[chapter]
\newtheorem{lemma}{\hspace{2em}引理}[chapter]
%\newtheorem{lemma}[definition]{引理}
\newtheorem{theorem}{\hspace{2em}定理}[chapter]
\newtheorem{axiom}{\hspace{2em}公理}[chapter]
\newtheorem{corollary}{\hspace{2em}推论}[chapter]
\newtheorem{exercise}{\hspace{2em}习题}[chapter]
\theoremsymbol{$\blacksquare$}
\newtheorem{example}{\hspace{2em}例}[chapter]

\theoremstyle{nonumberplain}
\theoremheaderfont{\CJKfamily{hei}\rmfamily}
\theorembodyfont{\normalfont \rm \CJKfamily{song}}
\theoremindent0em \theoremseparator{\hspace{1em}}
\theoremsymbol{$\blacksquare$}
\newtheorem{proof}{\hspace{2em}证明:}

%=========================== 修改引用的格式 ==============================%
% 第一行在引用处数字两边加方框
% 第二行去除参考文献里数字两边的方框
%\makeatletter
%\def\@cite#1{\mbox{$\m@th^{\hbox{\@ove@rcfont[#1]}}$}}
%\renewcommand\@biblabel[1]{#1}
%\makeatother
% 增加 \upcite 命令使显示的引用为上标形式
%\newcommand{\upcite}[1]{$^{\mbox{\scriptsize \cite{#1}}}$}             % 方法1
\newcommand{\upcite}[1]{\textsuperscript{\textsuperscript{\cite{#1}}}}  % 方法2


%=============================== 脚注 =============================%
\renewcommand{\thefootnote}{\arabic{footnote}}
%detcounter{footnote}{0}

%==================== 定义题头格言的格式 ==========================%
% 用法 \begin{Aphorism}{author}
%         aphorism
%      \end{Aphorism}
\newsavebox{\AphorismAuthor}
\newenvironment{Aphorism}[1]
{\vspace{0.5cm}\begin{sloppypar} \slshape
\sbox{\AphorismAuthor}{#1}
\begin{quote}\small\itshape }
{\\ \hspace*{\fill}------\hspace{0.2cm} \usebox{\AphorismAuthor}
\end{quote}
\end{sloppypar}\vspace{0.5cm}}

%============================== 控制表格线宽 ==========================%
% 更改横线(\hline)线宽:定义如下命令\hlinewd代替\hline。
% 更改垂直线(\vline)线宽:使用\usapackage{array},则可以在指定垂直线的地方用
% “!{\vrule width 3.5pt}”代替“|”,如“|c!{\vrule width 5pt}p{5cm}|r|”

\makeatletter
\def\hlinewd#1{%
  \noalign{\ifnum0=`}\fi\hrule \@height #1 \futurelet
   \reserved@a\@xhline}
\makeatother
\newcommand\vlinewd[1][1pt]{\vrule width #1}

% 不过上面的命令\hlinewd不能与longtable正常工作(reported by %钟圣俊老师),
% 只能使用下面的方法实现线宽控制:
%
%\setlength{\arrayrulewidth}{0.5pt}
%\setlength{\doublerulesep}{\arrayrulewidth}
%\newcommand{\dhline}{\hline\hline}
%\newcommand{\thline}{\hline\hline\hline}
%(类似的可以定义更多不同宽度的\hline)


%========================== 其它自定义 ==============================%
%====================================================================%
% 下面定义的命令(\alpheqn \reseteqn)可以使公式编号变为 4-a,4-b
% 使用说明:\alpheqn 为开始产生处,\reseteqn为恢复原来公式编号形式处
% 这两个命令为自定义,使用时应注意:不可放于 数学环境中!!!
% 在公式开始前和结束后使用!!!
%====================================================================%
\newcounter{saveeqn}%

\newcommand{\alpheqn}{%
\setcounter{saveeqn}{\value{equation}}%
\stepcounter{saveeqn}%
\setcounter{equation}{0}%
%\renewcommand{\theequation}{\arabic{saveeqn}-\alph{equation}}}%%article 中的定义
\renewcommand{\theequation}{\arabic{chapter}-\arabic{saveeqn}\alph{equation}}}%book %中的定义
%{\mbox{\arabic{equation}-\alph{equation}}}}%

\newcommand{\reseteqn}{%
\setcounter{equation}{\value{saveeqn}}%
%%\renewcommand{\theequation}{\arabic{equation}}}    %article 中的定义
\renewcommand{\theequation}{\arabic{chapter}-\arabic{equation}}}  %book 中的定义

%====================================================================%
% 下面定义的命令(\alphfig \resetfig)可以使插图编号变为 4-a,4-b
% 使用说明:\alphfig 为开始产生处,\resetfig为恢复原来插图编号形式处
% 这两个命令为自定义,使用时应注意:不可放于 数学环境中!!!
% 在插图开始前和结束后使用!!!
%====================================================================%
\newcounter{savefig}%

\newcommand{\alphfig}{%
\setcounter{savefig}{\value{figure}}%
\stepcounter{savefig}%
\setcounter{figure}{0}%
%%\renewcommand{\thefigure}{\arabic{savefig}-\alph{figure}}}%%article 中的定义
\renewcommand{\thefigure}{\arabic{chapter}-\arabic{savefig}\alph{figure}}}%book 中的定义
%{\mbox{\arabic{figure}-\alph{figure}}}}%

\newcommand{\resetfig}{%
\setcounter{figure}{\value{savefig}}%
%%\renewcommand{\thefigure}{\arabic{figure}}}    %article 中的定义
\renewcommand{\thefigure}{\arabic{chapter}-\arabic{figure}}}  %book 中的定义

%====================================================================%
% 下面定义的命令(\alphtab \resettab)可以使表格编号变为 4-a,4-b
% 使用说明:\alphtab 为开始产生处,\resettab为恢复原来表格编号形式处
% 这两个命令为自定义,使用时应注意:不可放于 数学环境中!!!
% 在表格开始前和结束后使用!!!
%====================================================================%
\newcounter{savetab}%

\newcommand{\alphtab}{%
\setcounter{savetab}{\value{table}}%
\stepcounter{savetab}%
\setcounter{table}{0}%
%%\renewcommand{\thetable}{\arabic{savetab}-\alph{table}}}%%article 中的定义
\renewcommand{\thetable}{\arabic{chapter}-\arabic{savetab}\alph{table}}}%%book 中的定义
%{\mbox{\arabic{table}-\alph{table}}}}%

\newcommand{\resettab}{%
\setcounter{table}{\value{savetab}}%
%%\renewcommand{\thetable}{\arabic{table}}}    %article 中的定义
\renewcommand{\thetable}{\arabic{chapter}-\arabic{table}}}  %book 中的定义

%====================================================================%
% 自定义项目列表标签及格式 \begin{mylist} 列表项 \end{mylist}
%====================================================================%
\newcounter{newlist} %自定义新计数器
\newenvironment{mylist}[1][可改变的列表题目]{%%%%%定义新环境
\begin{list}{\textbf{\hei #1} \arabic{newlist}:} %%标签格式
    {
    \usecounter{newlist}
     \setlength{\labelwidth}{22pt} %标签盒子宽度
     \setlength{\labelsep}{0cm} %标签与列表文本距离
     \setlength{\leftmargin}{0cm} %左右边界
     \setlength{\rightmargin}{0cm}
     \setlength{\parsep}{0.5ex plus0.2ex minus0.1ex} %段落间距
     \setlength{\itemsep}{0ex plus0.2ex} %标签间距
     \setlength{\itemindent}{44pt} %标签缩进量
     \setlength{\listparindent}{22pt} %段落缩进量
    }}
{\end{list}}%%%%%
