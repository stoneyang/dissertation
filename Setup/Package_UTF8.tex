%# -*- coding:utf-8 -*-
%=========================================================================%
%   LaTeX File for phd thesis of Institute of Automation, CAS
%-------------------------------------------------------------------------%

%=========================================================================%
%                          引用的宏包和相应的定义
%=========================================================================%

%============================ 中文支持宏包 ==============================%
% \usepackage{CJK}
\usepackage{CJKutf8}

%====================== 图形和超链接支持宏包 ========================%
% 图形支持宏包 为了使用PDFLaTeX需要作相应判断
% hyperref
% 可以在转换为pdf文件时自动加入索引, dvi2ps 加入-z 参数
% 为了支持pdftex 使用不同的宏包选项
\ifx\pdfoutput\undefined
\usepackage[dvips]{graphicx}
\graphicspath{
    {Figures/chap01/}
    {Figures/chap02/}
    {Figures/chap03/}
    {Figures/chap04/}
    {Figures/chap05/}
    {Figures/chap06/}
    {Figures/chap07/}
    {Figures/chap08/}
    {Figures/chap09/}
    {Figures/chap10/}
    {Figures/misc/}
}   %定义所有的eps文件在 Figures 子目录下
\usepackage[dvips,
            %pagebackref,
            backref,
            unicode=true,
            % CJKbookmarks=true,
            bookmarksnumbered=true
            % bookmarksopen=false,
            ]{hyperref}
\else
\usepackage[pdftex]{graphicx}
\graphicspath{
    {Figures/chap01/}
    {Figures/chap02/}
    {Figures/chap03/}
    {Figures/chap04/}
    {Figures/chap05/}
    {Figures/chap06/}
    {Figures/chap07/}
    {Figures/chap08/}
    {Figures/chap09/}
    {Figures/chap10/}
    {Figures/misc/}
}   %定义所有的eps文件在 Figures 子目录下
\usepackage[pdftex,
            %pagebackref,
            backref,
            unicode=true,
            % CJKbookmarks=true,
            bookmarksnumbered=true
            ]{hyperref}
\fi


% \usepackage{ccmap}


\usepackage{color}      % 支持彩色
\usepackage{subfig}     % 支持子图
\usepackage{floatflt}   % 图文混排用宏包
\usepackage{rotating}   % 图形和表格的控制
\usepackage{flafter}    % 因为图形可浮动到当前页的顶部,所以它可能会出现
                        % 在它所在文本的前面. 要防止这种情况,可使用 flafter
                        % 宏包

\usepackage[below]{placeins}    %浮动图形控制宏包
                                %允许上一个section的浮动图形出现在下一个
                                %section的开始部分该宏包提供处理浮动对象
                                %的 \FloatBarrier 命令,使所有未处理的浮动
                                %图形立即被处理
%\usepackage{endfloat}  %可将浮动对象放置到文件的最后
%\usepackage{overpic}   %将LaTeX对象放置在图上
%\usepackage{pstricks}  %Postscript macrosfor Generic TeX(我没用过,据说很强)
\usepackage{bez123}

\usepackage{tikz}
\usetikzlibrary{arrows,shapes}

%============================表格支持宏包=================================%
\usepackage{rotating}   % 用法 \begin{sidewaystable}....\end{sidewaystable}
                        % 即可旋转表格
\usepackage{longtable}  % 支持长表格
\usepackage{tabls}
\usepackage{multirow}   % 表格多行合并, 矩阵的边注
\usepackage{colortbl}   % 彩色表格
\usepackage{dcolumn}    % 让表格中将小数点对齐
\usepackage{hhline}     % 在表格中用 \hhline 得到的结果就如同\hline
                        % 或 \hline\hline,当然在和垂直线的交叉处会有所不同。
\usepackage{slashbox}   % 可在表格的单元格中画上一斜线。

%\newcommand{\centpcol}{\leftskip\fill \rightskip\fill}%制表使可用p{ncm}设置栏宽,还使本栏居中

%\iffalse 举例
%  \begin{tabular}{|l|c|c|} \hline
%    & \multicolumn{2}{c|}{MRD} \\ \cline{2-3}
%   & \multicolumn{1}{p{3.5cm}|}{\centpcol Same as previous response} &
%     \multicolumn{1}{p{3.5cm}|}{\centpcol Different from previous response}\\
%   \hline
%    Observed frequency & 966 & 148 \\
%    Expected frequency & 1066 & 48 \\ \hline
%    \multicolumn{3}{c}{${\chi}^2 = 213.97,\quad \mathit{d.f.}=1,\quad p<0.001$}
%  \end{tabular}
%\fi

%============================版面控制宏包=================================%
\usepackage[top=3.0cm,bottom=3.3cm,left=3.2cm,right=3.2cm, headheight=0.5cm, headsep=1.0cm, footskip=1.5cm %,includehead,includefoot
            ]{geometry}                 % 页面设置
\usepackage{indentfirst}                % 首行缩进宏包
\usepackage[perpage,symbol]{footmisc}   % 脚注控制
\usepackage{fancyhdr}                   % fancyhdr宏包 页眉和页脚的相关定义
\usepackage{lastpage}                   %自动记录总页数宏包,计数器为 LastPage
                                        %注意大小写
\usepackage{pageno}     % 章首页的页眉处理, 可以改为自己想要的形式
%\iffalse \makeatletter
%\renewcommand{\ps@plain}{%
%   \renewcommand{\@mkboth}{\@gobbletwo}%
%   \renewcommand{\@evenhead}{\reset@font\sf -- \thepage -- \hfil}%
%   \renewcommand{\@oddhead}{\reset@font\sf\hfil -- \thepage --}%
%   \renewcommand{\@evenfoot}{}%
%   \renewcommand{\@oddfoot}{}
%} \makeatother \fi

%======================= 多列文本与多列编号宏包 ===========================%
\usepackage{multicol,multienum}
%\iffalse 用法为:可嵌套使用
%\begin{multienumerate}[evenlist,oddlist]
%\mitemxxxx{Not}{Linear}{Not}{Quadratic}
%\mitemxxxo{Not}{Linear}{No; if $x=3$, then $y=-2$.}
%\mitemxx{$(x_1,x_2)=(2+\frac{1}{3}t,t)$ or
%$(s,3s-6)$}{$(x_1,x_2,x_3)=(2+\frac{5}{2}s-3t,s,t)$}
%\end{multienumerate}
%\begin{multicols}{2}
%\end{multicols}
%\fi

%======================== 数学公式相关宏包 ===============================%
\usepackage{bm}         % 处理数学公式中的黑斜体的宏包
\usepackage{amsmath}    % AMSLaTeX宏包 用来排出更加漂亮的公式
\usepackage{amssymb}    % AMSLaTeX宏包 用来排出更加漂亮的公式
\usepackage{mathrsfs}  % 不同于\mathcal or \mathfrak 之类的英文花体字体
\usepackage[amsmath,thmmarks]{ntheorem} % 定理类环境宏包,其中 amsmath 选项
                                        % 用来兼容 AMS LaTeX 的宏包
\usepackage{subeqnarray} %多个子方程(1-1a)(1-1b)
%\iffalse 以下是一个例子
%\begin{subeqnarray}
%\label{eqw} \slabel{eq0}
% x & = & a \times b \\
%\slabel{eq1}
% & = & z + t\\
%\slabel{eq2}
% & = & z + t
%\end{subeqnarray}
%\fi

\usepackage{algorithm}
\usepackage{algorithmic}

%改成后面的小标题
%\renewcommand{\algorithmicrequire}{\textbf{Initialization:}}   
\renewcommand{\algorithmicensure}{\textbf{Iteration:}}
%\renewcommand{\algorithmiclastcon}{\textbf{Output:}}

%=============================标题与列表宏包=============================%
\usepackage[sf]{titlesec}   % 控制标题的宏包,配合命令在后面,
                            % 将cjk+miktex+scrbook+gb.cap下的章的标题号,
                            % 比如~``第二章 XXX''位置于中心
\usepackage{cite}           % 支持引用的宏包
\usepackage{enumerate}      % 改变列表标号样式宏包 其后可接选项[a,A,i,I,1]
\usepackage{caption}        % 浮动图形和表格标题样式
%\usepackage{caption2}      % 浮动图形和表格标题样式,可选项为
                            % [scriptsize,footnotesize,centerlast]
%\usepackage{setspace}      % 图形和表格的标题如果是多行,行距比较大,可以加宏包
\usepackage{pifont}         % 有很漂亮的带圈的各种数字符号使用
\usepackage{Setup/atbeginend}    % 可选宏包, 能解决许多问题,
                            % 比如itemize, enumerate环境\item之间的控制
%用法
\AfterBegin{itemize}{\addtolength{\itemsep}{-0.5\baselineskip}}
\AfterBegin{enumerate}{\addtolength{\itemsep}{-0.5\baselineskip}}

%=================== 支持LaTexCAD生成的源程序的宏包 =====================%
\usepackage{Setup/TEXcad/lgrind}     % for formated source code
\usepackage{Setup/TEXcad/latexcad}   % latexcad.sty for drawings etc
\usepackage{Setup/TEXcad/epic}       %picture macros
\usepackage{Setup/TEXcad/eepic}      %extended picture macros
\usepackage{Setup/TEXcad/fancybox}   %box macros
\usepackage{Setup/TEXcad/epsf}       %postscript macros
\usepackage{Setup/TEXcad/rotate}     % postscript text rotation macros


%=========================== 特殊文本元素宏包 ==============================%
\usepackage{nicefrac}   % 在正文文本中排版分式时,可以用它来得到较好的排版效果。
\usepackage{units}      % 基于 nicefrac 宏包,提供对计量单位比较美观的排版效果。
\usepackage{soul}       % 支持对单词加上下划线或其每个字母在一定的宽度内均匀散布
\usepackage{altfont}    % 使用该宏包, 可以在一个宏包中使用多种不同的字体,
                        % 包括PSNFSS 和 MFNFSS
%\usepackage{prelim2e}  % 可以在每页页脚下方标记出本文档的版本信息等
\usepackage{a0size}     % 自由定义字号,在后面“字号设置”中可设置到107pt为止的大字体
%\usepackage{ulem}      % 用于支持文字内容的删除(非latex源文件中的注释!)(对中文的支持不够好,不会换行)
\usepackage{CJKulem}    % 用于支持文字内容的删除(非latex源文件中的注释!)(CJK字体专用)

%=========================== 其它 ==============================%
\usepackage[square,sort&compress]{natbib}
